\documentclass[12pt, 
    twoside=false, 
    bibliography=totoc, 
    numbers=endperiod, 
    headings=normal, 
    toc=chapterentrydotfill
    ]{scrbook}
\usepackage[T1]{fontenc}
\usepackage[utf8]{inputenc}
\usepackage[ngerman]{babel}
\usepackage{blindtext}
\usepackage{csquotes}
\usepackage{booktabs}
\usepackage{setspace}
\usepackage{mathpazo}
\usepackage{graphicx}
\usepackage{etoolbox}
\usepackage[
  	pdfstartview=FitH,   
  	pdffitwindow=true,
  	colorlinks,
  	linkcolor=black,
  	anchorcolor=black,
  	citecolor=black,
  	urlcolor=black
  	]{hyperref}
\usepackage[labelfont=bf]{caption}
\usepackage{float}
\usepackage[
    backend=biber, 
    style=authoryear-ibid, 
    eprint=false,
    url=false,
    doi=false,
    isbn=false,
    dashed=false
    ]{biblatex}
\addbibresource{Bib.bib}

% Layoutanpassungen
\setkomafont{sectioning}{\normalcolor\bfseries}
\renewcommand*{\chapterheadstartvskip}{\vspace*{-\topskip}}
\KOMAoptions{headsepline = true}
\setlength{\textheight}{1.05\textheight}
\AtBeginEnvironment{quote}{\singlespacing\small}

\begin{document}

\begin{titlepage}
    \begin{minipage}[t]{0.6\textwidth}
    \flushleft 
    Universität Hamburg \\
    Fachbereich: Sozialwissenschaften \\
    Fachgebiet: Politikwissenschaft \\
    Seminar: Forschungsseminar Vergleichende und Regionalstudien \\ 
    Dozenten: Prof. Kai-Uwe Schnapp \\
    PD Dr. Falk Daviter \\
    Wintersemester 2018/19 \\
    \end{minipage}
    \hfill
    \begin{minipage}[t][1.7cm][b]{0.35\textwidth}
    \includegraphics[width=\textwidth]{images/UHH-Logo_2010_Farbe_CMYK.pdf}
    \end{minipage}
    
    \vspace*{\fill}
    \begin{center}
	\vspace{1cm}\noindent {\textbf{Projektarbeit}} \vspace{0.2cm} \\
	\textbf{\Large Geschlechtsbezogene Partizipationsunterschiede in Parlamentsdebatten \\
	\vspace {0,5cm} \small\emph{Inwieweit unterscheiden sich Redebeiträge und Verhalten von weiblichen und männlichen Abgeordneten im 19. Deutschen Bundestag bezüglich Häufigkeit, Thematik und Geschlechterneutralität}} \\
	\vspace{0.5cm}
	30.03.2019
	\end{center}
    \vspace*{\fill}
	
	\begin{minipage}[t]{0.48\textwidth}
    \flushleft 
    Gina-Gabriela Görner \\
    Matrikelnummer: ***REMOVED*** \\
    XXX \vspace{0.1cm} \\ 
	XXX \vspace{0.1cm}  \\
	E-Mail: Gina-Gabriela.Goerner@Studium.Uni-Hamburg.de \\ 
    \end{minipage}
    \begin{minipage}[t]{0.48\textwidth}
	\flushleft
	Josef Holnburger \\
	Matrikelnummer: XXX \\
	XXX \vspace{0.1cm} \\
	XXX \vspace{0.1cm} \\
	E-Mail: josef@holnburger.com \\
    \end{minipage}

\end{titlepage}

\frontmatter

\tableofcontents

\listoffigures
\addcontentsline{toc}{chapter}{\listfigurename}
\vspace*{24pt}
{\let\clearpage\relax \listoftables}	
\addcontentsline{toc}{chapter}{\listtablename}

\mainmatter

\setstretch{1.5}


\chapter{Einleitung}\label{Einleitung} 

am Ende schreiben... 

Wichtig zu Beginn: Ziel der vorligeneden Arbeit ist die Untersuchunge der folgenden 5 Ebenen - welche jeweils in einem differezierten Verhältnis zum Repräsentationsbegriff stehen: 
\begin{itemize}
    \item Ebene 1: Repräsentation allgemein im Parlament : Anwesend/Abwesend/ Wie viele Frauen sind anwesend? 
    \item Ebene 2: Repräsentation allgemein in der Debatte: Teilnahme/Keine Teilnahme /Wie viel Teilnahme? 
    \item Ebene 3: Repräsentation Stereotypisierungen in der Debatte: unterschiedliche Themen/ gleiche Themen / Stereotypsierungen erfüllt/ nicht erfüllt ? 
    \item Ebene 4: Repräsentation innerhalb der Debatte: Einbeziehung von Frauen ja/nein in Form von: GFL ja/nein 
    \item Ebene 5: Repräsentation gesellschaftlichen Verhaltensmustern auch im Parlament - mehr Unterbrechungen / negative Unterbrechungen bei Frauen  ja/nein 
\end{itemize}



\chapter{Theoretischer Hintergrund}

\section{Politik als Arbeitsperspektive} 
Dieses Kapitel muss noch überarbeitet werden!! 
Politics as a workplace ? (Kap orientiert an Erikson und Josefsson 2018) (alle genannte Lit ist bereits bei Zotero eingetragen - jedoch bisher noch nicht geprüft)  

Die bisherige Forschung zur politischen Repräsentation von Frauen fokussiert sich primär auf die deskriptiven Themen, wie etwa der Frage danach, wie gesetzgebende Gremien durch Quoten geschlechtergerechter werden können (vgl. Dahlerup and Freidenvall, 2005; Schwindt-Bayer, 2009 NOCH NICHT LIT geprüft aber Zotero hinzugefügt) oder auf substantive Themen, welche beispielsweise untersuchen, ob weibliche Gesetzgeberinnen einen positiven Einfluss auf genderfreundliche politische Ergebnisse haben (Vgl. Beckwith and Cowell-Meyers, 2007 LIT noch prüfen, schon hinzugefügt Zotero) [ alles in Erikson, Josefsson 2018:199] .

In Anlehnung an Erikson & Josefsson (2018) wurden den inneren Mechanismen der gesetzgebenden Körperschaften sowie der Frage, „how the political game itself is gendered“‘ (ebd.) weniger empirische Aufmerksamkeit gewidmet (siehe hierzu Childs, 2016; Dahlerup and Leyenaar, 2013; Wangnerud, 2015) (NOCH NICHT LIT geprüft aber Zotero hinzugefügt) [ alles in Erikson, Josefsson 2018:199].

Ebenso wie Erikson und Josefsson (2018) argumentieren wie, dass das legislative Arbeitsumfelds für sich genommen ebenso wichtig sind, wie die Möglichkeiten der weiblichen Gesetzgeberinnen, die Ergebnisse zu beeinflussen. Politikerinnen sollten in der Lage sein, ihre Aufgaben als Gesetzgeberinnen auf Augenhöhe mit ihren männlichen Kollegen erfüllen (Vgl. Erikson, Josefsson 2018:199). 

Dahlerup (1988, 2006 - Literatur noch nicht geprüft aber bei Zotero eingefügt)) unterscheidet zwischen zwei verschiedenen Perspektiven der substantiellen Repräsentation von Frauen. Zum einen existiert die politische Outcome-Perspektive, welche die wissenschaftliche Literatur zur substantiellen Repräsentation von Frauen tendenziell dominiert. Zum anderen existiert die Perspektive der Politik als Arbeitsperspektive, welche bisher weniger häufig diskutiert wurde (Dahlerup 2006:513, SIEHE IN Erikson und Josefsson 2018: 199). 

Während sich die Outcome-Perspektive darauf konzentriert, ob weibliche Gesetzgeberinnen den Inhalt politischer Entscheidungen beeinflussen, indem sie diese geschlechterfreundlicher gestalten oder eine feministische Agenda verfolgen, beschäftigt sich die zweite Perspektive mit den Möglichkeiten, wie weibliche MPs als Repräsentantinnen gleichberechtigt mit ihren männlichen Kollegen auftreten können (Dahlerup 1988, 2006 Literatur noch nicht geprüft aber bei Zotero eingefügt - SIEHE IN Erikson und Josefsson 2018: 199).

Der ‚workplace approach‘ (in Orientierung an Erikson und Josefsson 2018:199) zielt insgesamt darauf ab, die Repräsentation von Frauen aus einer breiteren Perspektive zu betrachten, indem der Fokus von den Ergebnissen auf die geschlechterspezifischen Bedingungen innerhalb der Legislative. verlagert wird. Nach Dahlerup (1988, 2006) betrachten wir, ebenso wie Erikson und Josefsson (2018:199), die Arbeitsbedingungen im deutschen Bundestag sowohl als wichtig und halten sie zudem für einen wesentlichen Bestandteil für die Möglichkeiten, wie Frauen die politischen Ergebnisse beeinflussen können (Erikson und Josefsson 2018:199). 

\section{Legislative Gremien als maskuline Organisation }


Orientiert an Erikson, Josefsson 2018:200 : Legislative bodies: Masculine and male-dominated organizations

-in Arbeit-

Innerhalb der legislativen Gremien wurde das Geschlechterregime ebenso wie in anderen von Männern dominierten Sektoren häufig als ‚permeated by a culture of masculinity‘ beschrieben (Lovenduski, 2005:  lit nicht geprüft aber in Zotero 48 und siehe in Erikson, Josefsson 2018:200). In Anlehnung an Acker 1990 zeigt sich dies beispielsweise in der Existenz von formalen Regeln, die von Männern geschaffen wurden und einer männlich dominierten Organisation angepasst sind. Außerdem in Normen, wie sich ein (männlicher) Politiker präsentieren und verhalten soll (vgl. Acker 1990 :  lit nicht geprüft aber in Zotero 48 und siehe in Erikson, Josefsson 2018:200). Frauen werden infolgedessen mit dieser bereits existierenden ‚culture of masculinity‘ konfrontiert, die als institutionelle Einschränkung fungieren kann, wenn ihre Arbeit dadurch behindert wird (Lovenduski 2005:47-56 nicht geprüft aber in Zotero 48 und siehe in Erikson, Josefsson 2018:200). Erikson und Josefsson 2018:200) verweisen auf sozialpsychologische Forschungen, wonach Frauen in männerdominierten Bereichen häufig diskriminiert werden weil es an Übereinstimmungen zwischen den männerdominierten beruflichen Normen und den Eigenschaften mit denen Frauen typischerweise assoziiert werden mangelt (siehe hierzu Burgess und Borgida 1999; Eagly und Karau 2002; Heilmann 2001; Heilmann et al. 2004 --- schon in Zot aber noch nicht geprüft !!!))). Frauen riskieren demnach Diskriminierungen, weil sie entweder als weniger kompetent angesehen werden oder weil sie weibliche Attribute verletzen, wenn sie sich den Normen anpassen, so Erikson und Josefsson (2018:200). Die Folge sind neben der Disqualifikation die Abwertung der Leistungen von Frauen sowie die ungleiche Behandlung von Frauen, einschließlich von Belästigungen (Erikson und Josefsson 2018:200, siehe auch Burgess und Borgida 1999, Heilman 2001 – nicht geprüft, schon in zot):

“The gendered consequences of masculine norms are often manifested in the first case in the disqualification of women or the devaluation of their performance, whereas they often take the form of a disparate treatment of women in the second, including harassment” (Erikson und Josefsson 2018:200) 
Informelle Praktiken und Normen können demnach als Hindernisse für die Schaffung eines geschlechtergerechten Arbeitsumfeldes angesehen werden (Erikson und Josefsson 2018:200). 



\chapter{Relevanz des Forschungsgegenstandes}

\begin{itemize}
\item - Gleichberechtigung
\item -  Frauen im Parlament – zunehmend in der öffentlichen Debatte … 
\item -  die Zahl der Frauen im Parlament zunehmend in den Vordergrund der wissenschaftlichen Auseinandersetzung und der Politik. 
\end{itemize}

\begin{quote}
    \enquote{While women have made substantive progress in their representation in politics, they are still well underrepresented in political life in most nations} \parencite[2]{coffe_2013}
\end{quote}

Wenngleich ein politischer Konsens darin besteht, dass Gleichheit - und impliziert Geschlechtergleichheit- eine fundamentale Komponente von Demokratie darstellt, gibt es noch immer einen Gender-Gap in weltweit jedem Parlament (NOCH MAL PRÜFEN - ob Ruanda).  .... Repräsentation von Frauen für Demokratie notwendig .... 
…
\begin{itemize}
\item Anne Phillips gilt als eine der wichtigsten Vertreterinnen auf dem Gebiet der parlamentarischen Repräsentativität von Frauen. In ihrem Werk \emph{politics of presence} \parencite*{phillips_1998} analyisiert Phillips unter anderem die Bereitschaft von Frauen, als Parlamentskandidatin ausgewählt zu werden, Ministerin zu werden und Wahlen zu gewinnen \parencite[vgl.][416f.]{blaxill_2016}. 
Weil Frauen im täglichen Leben andere Erfahrungen sammeln als Männer -- insbesondere bezüglich der Kindererziehung, Bildung sowie der Auswahl an Berufen und der Unterscheidung von bezahlter und unbezahlter Arbeit, Gewalterfahrungen und sexuellen Belästigungen -- plädiert Philips für die notwendige Repräsentation von Frauen im Parlament, um andere Frauen vertreten zu können \parencite[vgl.][52]{wangnerud_2009}.
\end{itemize}

\begin{quote}
    \enquote{There are particular needs, interests, and concerns that arise from women's experience, and these will be inadequately addressed in a politics that is dominated by men. Equal rights to a vote have not proved strong enough to deal with this problem; there must also be equality among those elected to once} \parencite[66]{phillips_1998}
\end{quote}

Bereits Piktin \parencite*{pitkin_1972} argumentierte in ihrem Hauptwerk "The Concept of Representation", dass der Interessenbegriff in der Repräsentationsdebatte omnipräsent ("ubiquitious") ist \parencite[69]{wangnerud_2000}. Für Pitkin sind es die Parlamentarierinnen, welche sich den Wünschen und Interessen, dem Wohlergehen sowie Themen der Frauen widmen und diese vertreten \parencites[vgl.][413]{blaxill_2016}{pitkin_1972}. 

In Anlehnung an Phillips Theorie der \emph{politics of presence} \parencite*{phillips_1998} wird angenommen, dass weibliche Politikerinnen die Interessen und Wünsche weiblicher Bürgerinnen besser als männliche Politiker repräsentieren können. Hierbei wird die deskriptive mit der substantiellen Repräsentation verknüpft \parencite[52]{wangnerud_2009} : ein höherer Frauenanteil in Parlamenten führt zu einer stärkeren Thematisierung der Belange von Frauen.

\begin{itemize}
\item Relevanz von Sprache:
\end{itemize}

Während die Forderungen einer deskriptiven parlamentarischen Geschlechtergleichheit bereits ihren Weg in die wissenschaftliche Diskussion und in die Politik gefunden haben, wurde der geschlechtergerechten Sprache sowie dem geschlechterbezogenen Verhalten in den Parlamenten bisher weniger Aufmerksamkeit gewidmet. 
Laut \textcite{menegatti_2017} ist die Sprache eine der einflussreichsten Faktoren, wodurch Sexismus und Geschlechterdiskriminierung gefördert und reproduziert werden \parencite*[1]{menegatti_2017}. Sprache kann hierbei insbesondere die sozialen Asymmetrien von Status und Macht zugunsten des Mannes reproduzieren (ebd.). Es existier[t]en einvernehmliche Normen, wonach der prototypische Mensch ein Mann ist, was sich in den Strukturen vieler Sprachen widerspiegelt und darin verankert ist. Viele grammatikalische und syntaktische Regeln sind so aufgebaut, dass weibliche Ausdrücke normalerweise von entsprechenden männlichen Formen abgeleitet werden (ebd.). Männliche Substantive und Pronomen werden beispielsweise häufig mit einer generischen Funktion verwendet um sich sowohl auf Männer als auch auf Frauen zu beziehen. Auf diese Weise verschwinden Frauen allerdings aus der mentalen Repräsentationen \parencites{vaughan_2018}{stahlberg_2001}. Maskuline Generika lassen Leser*innen und Hörer*innen mehr in männlichen als weiblichen Personenkategorien denken \parencites[2]{sczesny_2016}{stahlberg_2007}.

\begin{quote}
    \enquote{Given that language not only reflects stereotypical beliefs but also affects recipients’ cognition and behavior, the use of expressions consistent with gender stereotypes contributes to transmit and reinforce such belief system and can produce actual discrimination against women} \parencite[2]{menegatti_2017}
\end{quote}

\begin{itemize}
\item - CRITICAL MASS???  hier ggf noch 
\item - gendered -shelley et al würde ich rauslassen - was spricht dagegen? 
\item - Fehlt hier noch - Stereotypen als Theoretischer Unterbau und Unterbrechungen ...
\end{itemize}



\section {Derzeitiger Forschungsstand und relevante Studien }

In der Forschung zu Frauen im Parlament wird generell zwischen einer deskriptiven und einer substantiellen Repräsentation (in einigen Fällen zudem die symbolische Form der Repräsentation) unterschieden. Bei der substantiellen Repräsentation von Frauen handelt es sich um ein bisher weniger wissenschaftlich erforschtes Feld als die Analyse der deskriptiven Repräsentation \parencite[59]{wangnerud_2009}. 
Letzteres legt den Schwerpunkt auf die Analyse der Anzahl von Frauen in repräsentativen Institutionen, ersteres untersucht hingegen die Auswirkung der Präsenz von Frauen in Parlamenten \parencites[14]{coffe_2013}[52]{wangnerud_2009}.
Die zentralen Fragen lauten hierbei:

\begin{quote}
  \enquote{[W]hether the widely professed aspiration to feminise democracy -- and in so doing to politically empower women -- is a matter largely of symbolism or substance […]}
  \enquote{[…] wheter the priority should simply be to increase the proportion of women MPs in Parliament […] or as Hanna Pitkin argued in 1967, to represent minds as well as bodies.}
  \parencite[413]{blaxill_2016}
\end{quote}

%
Die zu Beginn erläuterte Theorie der \emph{politics of presence} \parencite{phillips_1998} diente Wängnerud \parencite*{wangnerud_2000} als Ausgangspunkt für ihre Forschung im Schwedischen Parlament (\enquote{Testing the Politics of Presence, Women's Representation in the Swedish Riksdag}, 2000). Wägnerud prüft hierbei die Hypothese, ob weibliche Politikerinnen die Interessen von Frauen stärker vertreten als männliche Politiker \parencite[84]{wangnerud_2000} und kommt zu folgendem Ergebnis: \enquote{[It is] difficult to repudiate the conclusion that women's interes are primarily represented by female politicians} \parencite[][84]{wangnerud_2000}. Laut Wängnerud \parencite*{wangnerud_2000} gibt es allerdings keine eindeutige Einschätzung darüber, welche Auswirkungen zu erwarten sind, wenn die Zahl der Frauen im Parlament steigt \parencite{wangnerud_2009}.
Die Forschung von \textcite{celis_2008} greift diese Annahme empirisch auf und untersucht, in welchem Ausmaß die Bedürfnisse und Interessen von Frauen gesteigert werden, wenn die Anzahl von Frauen in politischen Entscheidungen zunimmt \parencite[vgl. auch][4]{galligan_2016}. Ein einfacher Anstieg der Anzahl von Frauen reicht nach \textcite{celis_2008}  nicht aus, um einen signifikanten Einfluss zu erreichen. In Anlehnung an Caul \parencite*{caul_2001}  sei  es beispielsweise notwendig, dass weibliche Abgeordnete in einflussreichen, wichtigen Positionen vertreten sind, um substantielle Repräsentation zu ermöglichen \parencites{caul_2001}[vgl. auch][14]{coffe_2013}.
Auch die Forschung von \textcite{back_2014} fragt nach den Auswirkungen, wenn die Zahl der Frauen im Parlament steigt und kommt zu folgendem Ergebnis: 

\begin{quote}
  \enquote{The variation found here is, however, not in line with the hypothesis suggesting that
increased descriptive representation should lead to a rise in the substantive representation so that the prediction drawn from the critical mass theory is not given support here. Instead, we find that women are more underrepresented in legislative debates when they represent parties with many female MPs.}
  \parencite[17]{back_2018}
\end{quote}


\textcite{back_2018}  untersuchen länderübergreifend sieben europäische Länder bezüglich der Themen und Redeanteile weiblicher Abgeordneter und resümieren, dass Frauen in Parlamenten seltener das Wort ergreifen. Dieses Ergebnis ist nicht auf einen generell niedrigeren Anteil an Frauen in den Parlamenten zurückzuführen, so \textcite{back_2018}. Frauen in Parteien mit einem geringen Frauenanteil ergreifen laut der Untersuchung von Bäck und Debus sogar häufiger das Wort als Frauen in Fraktionen mit einem hohen Anteil an weiblichen Mitgliedern \parencite[17]{back_2018}. 

\begin{itemize}
\item Backlash?????? Thematisieren!!!!
\end{itemize}

In einer früheren Studie konnte zudem festgestellt werden, dass außerdem Geschlechterunterschiede bezüglich der Themen der Reden festzustellen sind \parencite[514f.]{back_2014}. \textcite{back_2014} konnten in ihrer Untersuchung des schwedischen Parlaments nachweisen, dass Männer im schwedischen \emph{Riksdag} häufiger zu sogenanntenn \emph{hard policies} (etwa Wirtschaft, Energie, Infrastruktur) sprechen. Bei sogenannten \emph{soft policies} (Erziehung, Soiale Sicherung, Bildung) ist hingegen kein Unterschied bezüglich des Redeanteils und Geschlecht der Abgeordneten festzustellen \parencite[514f.]{back_2014}. 

\begin{itemize}
\item Es konnte gezeigt werden, dass…… 


\item Eine umfassende Untersuchung von Geschlechterunterschieden in Bezug auf Sprache, Thematik sowie Verhalten der Parlamentsmitglieder des Deutschen Bundestages ist allerdings bis zu diesem Zeitpunkt noch ausstehend, wenngleich erforderlich: 
\end{itemize}

\citereset
% Mit einem citereset setzen wir (ebd.) zurück. Ist notwendig, wenn wir Quoten, da Blockquotes nicht mit ebd. geführt werden sollen.
\begin{quote}
 \enquote{Our research clearly suggest that gender plays a role in parliamentary speech-making and the selection of the MPs who take the parliamentary floor, which calls for further comparative research on the role of gender in legislative  debates in different institutional contexts and with varying degrees of descriptive representation}
  \parencite[515]{back_2014}
\end{quote}


!!!!!
NOCH EINARBEITEN, umformulieren und Lit prüfen:
In Bezug auf die Interaktion zwischen GesetzgeberInnen konnten frühere Untersuchungen zeigen, dass Frauen und Männer innerhalb der gesetzgebenden Gremien nicht nur unterschiedliche Politik betreiben, sondern außerdem unterschiedlich behandelt werden (Childs 2004 lit in zot aber noch nicht geprüft, SIEHE in Erikson und Josefsson 2018: 201). Zudem gibt es Hinweise darauf, dass männliche Gesetzgeber bei einem zunehmenden Frauenanteil im Parlament, unhöflich und respektlos gegenüber weiblichen Gesetzgeberinnen sein können (Kathlene 1994, 2004 lit in zot aber noch nicht geprüft, SIEHE in Erikson und Josefsson 2018: 201) und dass weibliche Gesetzgeberinnen offener Diskriminierungen und sexueller Belästigung ausgesetzt sind (Lovenduski 2005:76 2004 lit in zot aber noch nicht geprüft, SIEHE in Erikson und Josefsson 2018: 201))




\chapter{Übergeordete Forschungsfrage und Konzeption der Arbeit}

\begin{itemize}
\item Während sich eine Vielzahl der wissenschaftlichen Forschung mit der Partizipation und Repräsentation von Frauen in Parlamenten aus einer deskriptiven und substantiellen Perspektive auseinandersetzt, fokussiert die vorliegende Forschungsarbeit die geschlechterspezifischen Unterschiede der Parlamentsdebatten.
Neben dem Anteil an Reden von Frauen im Deutschen Bundestag soll insbesondere inhaltlich auf die Reden des Deutschen Bundestages eingegangen werden und das Verhalten der Abgeordneten in Bezug auf geschlechtsspezifische Besonderheiten untersucht werden. Die zentrale Forschungsfrage lautet: \enquote{\emph{Inwieweit unterscheiden sich die Redebeiträge und Verhalten von weiblichen und männlichen Abgeordneten im 19. Deutschen Bundestag bezüglich Häufigkeit, Thematik und Geschlechterneutralität}}.
\end{itemize}





\section{Hypothese Genderinklusive Sprache}


Es wird im folgenden davon ausgegangen, dass Sprache eine der einflussreichsten Faktoren ist, wodurch Sexismus und Geschlechterdiskriminierung gefördert und reproduziert werden \parencite*[1]{menegatti_2017}. Es existier[t]en einvernehmliche Normen, wonach der prototypische Mensch ein Mann ist, was sich in den Strukturen vieler Sprachen widerspiegelt und darin verankert ist. Sprache kann soziale Asymmetrien von Status und Macht zugunsten des Mannes reproduzieren \parencite{menegatti_2017}.
Männliche Substantive und Pronomen werden beispielsweise häufig mit einer generischen Funktion verwendet um sich sowohl auf Männer als auch auf Frauen zu beziehen. Auf diese Weise verschwinden Frauen allerdings aus der mentalen Repräsentationen \parencites{vaughan_2018}{stahlberg_2001}. Maskuline Generika lassen Leser*innen und Hörer*innen mehr in männlichen als weiblichen Personenkategorien denken \parencites[2]{sczesny_2016}{stahlberg_2007}.

\begin{quote}
    \enquote{Given that language not only reflects stereotypical beliefs but also affects recipients’ cognition and behavior, the use of expressions consistent with gender stereotypes contributes to transmit and reinforce such belief system and can produce actual discrimination against women} \parencite[2]{menegatti_2017}
\end{quote}

Die Verwendung von \emph{gender-fail linguistic} (GFL) kann diese negativen Auswirkungen effektiv verhindern und Geschlechtergerechtigkeit fördern \parencite[1]{menegatti_2017}. GFL wurde unter anderem im Rahmen eines umfassenden Versuchs zur Verringerung von Stereotypen und Diskriminierung in der Sprache eingeführt \parencite[2]{sczesny_2016}. Feminiserung und Neutralisierung sowie die Kombination beider sind die bevorzugten Strategien einer geschlechtergerechten oder geschlechterneutralen Sprache. Auch die Verwendung von Wortpaaren (Lehrerinnen und Lehrer) zählt ebenso zu den Ausdrucksformen der GFL. Neben Wortpaaren sind auch geschlechtsneutrale Formen (Studierende statt Student) mögliche GFL-Ausdrucksformen \parencite[2]{sczesny_2016}.

Aus den bereits genannten Problematiken ist es notwendig, die Sprachgewohnheit dahingehend zu ändern, GFL umfassend zu etablieren, um Vorurteile bezüglich der Geschlechter zu reduzieren und gleichzeitig eine Reproduktion von Stereotypen zu vermeiden. Die Verwendung von geschlechtergerechten Ausdrücken anstelle von maskulinen Generika ist für den Abbau von Geschlechtervoreingenommenheit und die Förderung der Gleichstellung der Geschlechter laut Menegatti und Rubini unabdingbar \parencite*{menegatti_2017}.

Die Umsetzung und Etablierung von GFL hat in verschiedenen Ländern bisher unterschiedliche Stadien erreicht und wird beispielsweise von der UNESCO und der Europäischen Kommission empfohlen und in deren Dokumenten angewandt \parencite[4]{sczesny_2016}.

\begin{quote}
    \enquote{[…] language does not merely reflect the way we think: it also shapes our thinking. If words and expressions that imply that women are inferior to men are constantly used, that assumption of inferiority tends to become part of our mindset; hence the need to adjust our language when our ideas evolve.} (UNESCO 2011) HINWEIS: UNESCO FEHLT NOCH IN ZOTERO GINA: nun hinzugefügt - UNSECO 2011
\end{quote}

Auf Basis der Argumentation von Wängnerud \parencites*{wangnerud_2000}{wangnerud_2009} wird erwartet, dass sich vor allem Frauen über die Ausgrenzungserfahrung durch Sprache bewusst sind oder entsprechende Erfahrungen bereits gemacht wurden und entsprechend eher eine geschlechterneutrale und geschlechtergerechte Sprache verwenden als Männer. 

\begin{quote}
    \enquote {Surprisingly, women changed their
language use more in the direction of gender-fair language than men.} \parencite[555]{koeser_2014}
\end{quote}

\begin{quote}
 \enquote {Because some individual factors such as sexism (Parks \& Roberton, 2008 - Lit kontrollieren) correlate with use of and attitudes toward gender-fair language, it is important to investigate their impact on the acceptance and rejection of such messages in more detail.} \parencite[556]{koeser_2014}
\end{quote} Parks \& Roberton, 2008 - Lit kontrollieren) 




\textbf{Hypothese 1:} \emph{Frauen verwenden in ihren Reden häufiger gender-fair language als Männer}


\section{Hypothese Themen der Reden}


\textcite{back_2014} konnten in ihrer Untersuchung des schwedischen Parlaments nachweisen, dass bezüglich der Themen der Reden Geschlechterunterschiede festzustellen sind. Männer sprechen laut Bäck et. al. im schwedischen \emph{Riksdag} häufiger zu sogenanntenn \emph{hard policies} (etwa Wirtschaft, Energie, Infrastruktur). Bei sogenannten \emph{soft policies} (Erziehung, Soiale Sicherung, Bildung) ist hingegen kein Unterschied bezüglich des Redeanteils und Geschlecht der Abgeordneten festzustellen \parencite[514f.]{back_2014}.

Die von \textcite{back_2014} vorgenommene Klassifizierung soll in unserer Forschung nicht vorgenommen werden, da hier die Gefahr besteht, in der Gesellschaft vorhandene Stereotypen zu reproduzieren, indem eine Klassifizierung in vermeintliche geschlechtsspezifische Themen vorgenommen wird.

Dennoch wird auf Basis der Argumentation von Wängnerud \parencites*{wangnerud_2000}{wangnerud_2009} und der Arbeit von \textcite{back_2014} untersucht, ob eine geschlechtsspezifische, unterschiedliche Thematisierung in den Reden feststellbar ist. 

\textbf{Hypothese 3:} \emph{Es sind unterschiedliche Thematisierungen in den Reden der Abgeordneten feststellbar.}


\section{Hypothese Unterbrechungen und Rückfragen in den Reden}


Die von Erikson und Josefsson JAHR!! durchgeführte Befragung zum Arbeitsumfeld von schwedischen Abgeordneten konnte zeigen, dass weibliche Abgeordnete mehr Stress und Druck in ihrem Umfeld ausgesetzt sind und häufiger Opfer negativer Behandlungen im Parlament sind als männliche Abgeordnete \parencite{erikson_2018}. Sie werden eher in den Debatten unterbrochen, sind eher Opfer sexistischer Witze und ihre Äußerlichkeiten wird häufiger negativ kommentiert als die Erscheinung männlicher Kollegen \parencite[13]{erikson_2018}.

Diese Mehrbelastung kann laut Erikson und Josefsson zu höheren "Kosten" für Frauen bezüglich des parlamentarischen Engagements führen - sie müssen mit mehr negativen Erfahrungen rechnen als Männer, entsprechend könnte dies auch ein Grund für eine geringere Beteiligung von Frauen in Parlamentsdebatten oder generell für die Partizipation von Frauen in politischen Institution darstellen \parencites[vgl.][]{erikson_2018}[vgl.][]{back_2014}.

Es wird erwartet, dass ein solches negatives Verhalten gegenüber Frauen auch im Deutschen Bundestag zu beobachten ist. Entsprechend soll untersucht werden, wie häufig Abgeordnete in ihren Reden \emph{negativ} unterbrochen werden. Entsprechend sollen Reden bezüglich der Anzahl an Zurufen, Gegenrufen, Widerspruch und Lachen über den*die Redner*in untersucht werden.\footnote{Diese Unterbrechungen werden in den Bundestagsprotokollen vermerkt. \emph{Lachen} kann hier als \emph{auslachen} oder \emph{lachen über den Inhalt} verstanden werden - das Lachen beispielsweise über einen Witz wird in Bundestagsprotokollen als \emph{Heiterkeit} protokolliert. \emph{Positive} Unterbrechungen wären beispielsweise Beifall.}

\textbf{Hypothese 2a:} \emph{Frauen werden häufiger als Männer während der Rede negativ Unterbrochen}

Eine weitere \emph{negative} Unterbrechung der Render*innen können Rückfragen darstellen -- allerdings ist hier eine eindeutige negative Intention nicht anzunehmen. Entsprechend sollen Rückfragen gesondert untersucht werden. Auf Basis der Arbeiten von \textcite{brescoll_2011} und \textcite{eagly_2002}  wird allerdings erwartet, dass Frauen bei Themen, welche stereotypisch eher nicht mit Frauen verbunden werden, häufiger Inkompetenz unterstellt wird als männlichen Abgeordneten. Dies kann sich in der Anzahl an Rückfragen äußern, welche während einer Rede gestellt werden. 

\textbf{Hypothese 2b:} \emph{Frauen erfahren häufiger Rückfragen in ihren Reden, wenn wenige andere Frauen zu diesem Thema gesprochen haben}\footnote{Diese Hypothese kann nur in Verbindung mit der nachfolgenden Hypothese zur Thematisierung in den Reden überprüft werden.}



\section{Weitere ergänzende Untersuchungen}

Da der 19. deutsche Bundestag bisher noch nicht Gegenstand einer Untersuchungen der unterschiedlichen Partizipation von Frauen und Männern in den Parlamentsdebatten war soll ebenso die Anzahl an Reden von weiblichen Abgeordneten nach Fraktion untersucht ermittelt werden. Wenngleich \textcite{back_2018} in einem Ländervergleich keine Verbindungen zwischen der Anzahl an Frauen in den Fraktionen und einer Anzahl der Reden in den Parlamenten aufzeigen konnte, soll diese Untersuchung des 19. deutschen Bundestag ergänzend in unsere Forschung aufgenommen werden. Auf diese Weise können die bisherigen Ergebnisse von \textcite{back_2018} überprüft werden.. 

\section{Kontrollvariablen}

\chapter{Datengrundlage und Methodik}

Seit der 19. Wahlperiode liegen die Protokolle des Deutschen Bundestags im TEI-Format (Text Encoding Initative)\footnote{http://www.tei-c.org/guidelines/} als XML-Dateien (Extensible Markup Language) vor. Abgerufen werden können sie über das /emph{Open Data Portal} des Deutschen Bundestags.\footnote{https://www.bundestag.de/service/opendata} In Verbindung mit den biografischen Daten aller Bundestagsabgeordneten\footnote{https://www.bundestag.de/blob/472878/e207ab4b38c93187c6580fc186a95f38/mdb-stammdaten-data.zip} können Aussagen über Anzahl und Inhalte der Reden nach Geschlecht, Alter, Fraktion und beispielsweise Anzahl der Bundestagsperioden getroffen werden.

\section{Vorgehen zur Auswertung der Bundestagsprotokolle}


Insgesamt liegen 5.965 Reden\footnote{Es liegen auch Reden der Regierungsmitglieder und Reden von Gästen vor, diese werden allerdings nicht ausgewertet, da hierbei keine Angaben zum Geschlecht der Redner*innen durch den Bundestag bereitsgestellt werden und zudem nur die Bundestagsdebatte und nicht die Berichterstattung durch die Regierung ausgewertet werden soll.} von Bundestagsabgeordneten mit Anmerkungen durch die Stenograf*innen vor. Diese Anmerkungen ermöglichen die Operationalisierung der Hypothesen -- so wird bei einer Rede auch ein Zwischenruf protokolliert und auch der*die Zwischenrufer*in im Protokoll vermerkt (wenn der Zwischenruf zugeordnet werden kann). Auch Angaben über die Tagesordnungspunkte (TOPs), zu denen gesprochen wird, sind im Protokoll vermerkt.

Die Daten werden mittels der Programmiersprache R \parencite{rcoreteam_2018} und den Paketen \emph{rvest} \parencite{wickham_2016}, \emph{xml2} \parencite{wickham_2018} und der Paketsammlung \emph{tidyverse} \parencite{wickham_2017} aufbereitet und ausgewertet.


Wie die Fragestellung \enquote{Inwieweit unterscheiden sich die Redebeiträge und Verhalten von weiblichen und männlichen
Abgeordneten im 19. Deutschen Bundestag bezüglich Häufigkeit, Thematik und Geschlechterneutralität} bereits suggeriert, wird davon ausgegangen, dass die Variable \emph{Geschlecht} einer von vielen Faktoren ist, welche das Verhalten, die Sprache und den Inhalt der Bundestagsdebatte beeinflussen. Entsprechend handelt es sich bei dieser Untersuchung um ein x-zentriertes Forschungsdesign, in welchem auch weitere Faktoren, welche auf die jeweilige abhängige Variable einwirken, untersucht werden sollen  \parencites[vgl.][3f.]{ganghof_2005}.

\section{Methodik Regression und automated topic modelling}


Die Überprüfung der \textbf{Hypothesen 1, 2a und 2b} sollen mittels Regressionsanalysen untersucht werden. Neben dem Geschlecht des*der Redner*in sollen folgende Faktoren als Kontrollvariablen erhoben und untersucht werden: (KRITIK HIERBEI; MUSS NACH OBEN; NICHT ERST HIER! GEHÖRT ZUR KONZEPTION)

\textbf{der*die Rednerin ist Mitglied der Oppositions- bzw. Regierungsfraktion}

Es wird erwartet, dass Mitglieder der Regierungsfraktion deutlich häufiger mit Rückfragen konfrontiert werden, da sie eine stärkere Einbindung in den Legislativprozess erfahren und die Kontrolle der Regierung zu den Kernaufgaben der Opposition zählen.

\textbf{der*die Redner*in ist Mitglied des Fraktionsvorsitzes}

Es wird erwartet, dass Mitglieder des Vorsitzes der Fraktionen beispielsweise häufiger unterbrochen werden. Zum einen, da Reden, welche auf Antrag der Opposition länger gehalten werden, häufig von den Fraktionsvorsitzenden gehalten werden, aber auch, da diese Reden häufig bei volleren Plenarsälen gehalten werden.

\textbf{der*die Redern*in ist Mitglied der Fraktion "Alternative für Deutschland"}

Hier wird erwartet, dass Abgeordnete der AfD-Fraktion deutlich seltener GFL nutzen, wohingehend die Anzahl der Unterbrechungen deutlich höher erwartet wird. Populistische Parteien setzen sich unter anderem durch provokante Thesen und ein provokantes Auftreten von anderen Parteien ab \parencites[vgl.][]{decker_populismus_2006}{priester_2012}. Dies lässt vermuten, dass mehr Zwischenrufe von anderen Abgeordneten in den Reden der AfD-Mitglieder geäußert werden.

\textbf{die Anzahl der bisherigen Bundestagsmandate des*der Redner*in}

Bei dieser Variable wird kein Einfluss auf die abhängigen Variablen (Nutzung von GFL, Anzahl der Unterbrechungen, Anzahl der Rückfragen) erwartet, sie dient der Überprüfung der Validität der Methoden.

\textbf{das Alter des*der Redner*in}

Hier wird keine Einfluss auf die abhängigen Variablen erwartet. Diese Variable dient der Überprüfung der Validität der Methode.

Die {Hypothesen 1, 2a und 2b} werden über eine Regressionsanalyse überprüft. Die Variable Geschlecht wird hierbei als Dummy-Variable kodiert. Der Einfluss der Kontrollvariablen etwa auf die abhängige Variable \emph{Anzahl der Unterbrechungen} wird ebenfalls überprüft. Da es sich bei den abhängigen Variablen um Zähldaten handelt, wird vermutlich eine negative Binomialregression angewandt (welche auch bei \textcite{back_2014} angewandt wurde) oder eine OLS-Regression untersucht (angewandt bei \textcite{coffe_2013}).


\textbf{Hypothese 3} zur Untersuchung der Themen in den Reden der Abgeordneten soll über ein automatisiertes \emph{Topic Modeling} untersucht werden. Anders als bei \textcite{back_2014}, welche die Themen der Reden der Abgeordneten anhand der vorangegangen Reden von Minister*innen untersucht hat, soll durch ein automatisiertes \emph{Topic Modeling} einerseits einen geschlechterstereotypen Bias verhindern aber auch eine genauere Zuordnung an Themen ermöglichen. 
Automatisierte \emph{Topic Modelings} wurden bereits auf Reden des Europaparlaments angewandt, um Agenda-Trends zu identifizieren \parencite[vgl.][]{greene_2016} und werden auch in der vergangenen Studie von Bäck und Debus zur Identifikation von geschlechtsbezogener \emph{Topics} vorgeschlagen \parencite*[18]{back_2018}.

Als automatisiertes \emph{Topic Modeling Framework} wird das von der University of Cambridge ausgezeichnete\footnote{https://www.cambridge.org/core/membership/spm/about-us/awards/statistical-software-award/statistical-software-award-18} \emph{Structural Topic Modeling} von \textcite{roberts_2018} angewendet. Dies ermöglicht es, auch innerhalb von Reden mehrere Inhalte zu identifizieren und erlaubt so ein genaueres Vorgehen bei der Untersuchung möglicher geschlechterbezogener Themen. In unserem Datensatz sind \textbf{XXX} Reden enthalten.

\subsection{Hintergrund stm}

\chapter{Ergebnisse und Auswertungen}

\section{Genderinklusive Sprache}

\begin{quote}
\enquote {The present study is the first one that investigated persuasion by arguments concerning gender-fair language. It indicates that arguments can be an effective tool for making speakers use gender-fair language.} \parencite[556]{koeser_2014}
\end{quote} 

\section{Themen der Reden}

\section{Unterbrechungen und Rückfragen}

\chapter{Methodenkriktik und Forschungsausblick}

\chapter{Fazit}

\setstretch{1.0}
\printbibliography[title={Literaturverzeichnis}]

\end{document}

\chapter{Beispiele LaTeX}

Hier eine Tabelle, welche im Ordner \enquote{table} abgespeichert wurde und mit folgendem Befehl eingelesen werden kann. Verweise siehe Tabelle \ref{table:unterbrechung_stats}. Außerdem möchte ich hier noch auf die Abbildung \ref{fig:lda_example} auf Seite \pageref{fig:lda_example} verweisen. 

\begin{table}[htb]
    \centering
    \caption{Statistiken zu Unterbrechungen in den Bundestagsreden}
    
\begin{tabular}{lrrrrrr}
\toprule
Geschlecht & min & max & mean & sd & n & var\\
\midrule
männlich & 0 & 41 & 3.964055 & 4.857574 & 4173 & 23.59602\\
weiblich & 0 & 39 & 3.276607 & 4.110263 & 1851 & 16.89426\\
\bottomrule
\end{tabular}
    \label{table:unterbrechung_stats}
\end{table}



\section{Unterkapitel}



\begin{figure}
    \centering
    \includegraphics[width=0.8\textwidth]{document/images/lda_topic_model.png}
    % Der Teil in eckigen Klammern ist der Kurztitel für das Table of Contents
    \caption[Schematische Darstellung eines LDA Topic Modelings]{Schematische Darstellung eines LDA Topic Modelings. Quelle: \citetitle{blei_2012} \parencite{blei_2012}}
    \label{fig:lda_example}
\end{figure}

% Wenn wir nicht wollen, dass ein Kapitel auf einer neuen Seite startet, nutzen wir folgenden Befehl hier:
{\let\clearpage\relax \chapter{Theoriefindung}}



%% Bibliography

\setstretch{1.0}
\printbibliography[title={Literaturverzeichnis}]

\end{document}