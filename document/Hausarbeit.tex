\documentclass[12pt, 
    twoside=false, 
    bibliography=totoc, 
    numbers=endperiod, 
    headings=normal, 
    toc=chapterentrydotfill
    ]{scrbook}
\usepackage[T1]{fontenc}
\usepackage[utf8]{inputenc}
\usepackage[ngerman]{babel}
\usepackage{blindtext}
\usepackage{csquotes}
\usepackage{booktabs}
\usepackage{array}
\usepackage{setspace}
\usepackage{mathpazo}
\usepackage{graphicx}
\usepackage{etoolbox}
\usepackage[
  	pdfstartview=FitH,   
  	pdffitwindow=true,
  	colorlinks,
  	linkcolor=black,
  	anchorcolor=black,
  	citecolor=black,
  	urlcolor=black
  	]{hyperref}
\usepackage[labelfont=bf]{caption}
\usepackage[draft]{todonotes} 
\usepackage{float}
\usepackage[
    backend=biber, 
    style=authoryear-icomp, 
    eprint=false,
    url=false,
    doi=false,
    isbn=false,
    dashed=false
    ]{biblatex}
\addbibresource{Bib.bib}

% Layoutanpassungen
\setkomafont{sectioning}{\normalcolor\bfseries}
\renewcommand*{\chapterheadstartvskip}{\vspace*{-\topskip}}
\KOMAoptions{headsepline = true}
\setlength{\textheight}{1.05\textheight}
\AtBeginEnvironment{quote}{\singlespacing\small}
\makeatletter
% Keine Seitenvorschübe bei \Chapter
\newcommand\Chapter{%
                    \par\vspace{1.5cm}% anpassen
                    \global\@topnum\z@
                    \@afterindentfalse
                    \secdef\@chapter\@schapter}
\makeatother
\DefineBibliographyStrings{ngerman}{
   andothers = {{et\,al\adddot}},
}

\begin{document}

\begin{titlepage}
    \begin{minipage}[t]{0.6\textwidth}
    \flushleft 
    Universität Hamburg \\
    Fachbereich: Sozialwissenschaften \\
    Fachgebiet: Politikwissenschaft \\
    Seminar: Forschungsseminar Vergleichende und Regionalstudien \\ 
    Dozenten: Prof. Kai-Uwe Schnapp \\
    PD Dr. Falk Daviter \\
    Wintersemester 2018/19 \\
    \end{minipage}
    \hfill
    \begin{minipage}[t][1.7cm][b]{0.35\textwidth}
    \includegraphics[width=\textwidth]{images/UHH-Logo_2010_Farbe_CMYK.pdf}
    \end{minipage}
    
    \vspace*{\fill}
    \begin{center}
	\vspace{1cm}\noindent {\textbf{Projektarbeit}} \vspace{0.2cm} \\
	\textbf{\Large Geschlechtsbezogene Repräsentationsunterschiede im Deutschen Bundestag \\
	\vspace {0,5cm} \small\emph{Inwiefern unterscheiden sich Redebeiträge und Verhalten von weiblichen und männlichen Abgeordneten im 19. Deutschen Bundestag bezüglich Häufigkeit, Thematik und Geschlechterneutralität}} \\
	\vspace{0.5cm}
	30.04.2019
	\end{center}
    \vspace*{\fill}
	
	\begin{minipage}[t]{0.48\textwidth}
    \flushleft 
    Gina-Gabriela Görner \\
    Matrikelnummer: 6436971 \\
    Friedensallee 15 \vspace{0.1cm} \\ 
	22765 Hamburg \vspace{0.1cm}  \\
	\href{mailto:gina-gabriela.goerner@uni-hamburg.de}{gina-gabriela.goerner@uni-hamburg.de} \\
    % Hab das geändert, e-mails schreibt man idR klein, wie webseiten. Sind nicht case-sensitiv
    \end{minipage}
    \begin{minipage}[t]{0.48\textwidth}
	\flushleft
	Josef Holnburger \\
	Matrikelnummer: 6524900 \\
	Beuthstraße 1 \vspace{0.1cm} \\
	10117 Berlin \vspace{0.1cm} \\
	\href{mailto:josef@holnburger.com}{josef@holnburger.com} \\
    \end{minipage}

\end{titlepage}


\tableofcontents
\thispagestyle{empty}

\frontmatter

\listoffigures
\addcontentsline{toc}{chapter}{\listfigurename}
\vspace*{24pt}
{\let\clearpage\relax \listoftables}	
\addcontentsline{toc}{chapter}{\listtablename}

\mainmatter

\setstretch{1.5}


\chapter{Einleitung}\label{Einleitung} 

\begin{quote}
    \enquote{Gender equality in political representation may be described within both optimistic and pessimistic stories.} \parencite[149]{celis_2018}
\end{quote}

\noindent
Die machtpolitische Gleichstellung der Geschlechter sowie die politische Ressourcenverteilung in den industrialisierten Demokratien ist in den letzten fünfzig Jahren enorm gewachsen \parencite[318]{coffe_2010}\todo{quelle mit vornamen angezeigt?}. Es gibt mehr weibliche Parlamentarierinnen (aus dem englischen \emph{Members of Parliament} = MPs) als je zuvor und eine Rekordzahl von Frauen nimmt Führungspositionen in den nationalen Regierungen \parencites{lovenduski_2005}{paxton_2007}, mit vielen wichtigen Konsequenzen für politische Ergebnisse und Prioritäten ein \parencites{bolzendahl_2007}{carroll_2001}{waring_2000}[318]{coffe_2010}.\todo{wieso c. bei bolzendahl davor??} Dennoch sind derzeit weltweit weniger als ein Viertel Frauen (24,3 Prozent) in den Parlamenten vertreten \parencite{ipu_2019}. Repräsentative Institutionen werden trotz eines langen Kampfes um Gleichstellung noch immer von Männern dominiert \parencites[149]{celis_2018}[497 f.]{childs_2013}{dahlerup_2013} {bjarnegard_2013} und es ist nicht umstritten zu behaupten, dass eine Gleichstellung der Geschlechter nicht erreicht wurde \parencite[150]{celis_2018}. Eine Gleichstellung wird hierbei nicht nur über die Anzahl der Sitze in repräsentativen Parlamenten definiert, sondern ergibt sich durch eine Vielzahl an Faktoren. Hierzu zählen unter anderem die sprachliche Einbeziehung aller Geschlechter\footnote{In der Literatur herrscht ein Minimalkonsens darüber, dass es sich bei Geschlecht um eine soziale Konstruktion handelt \parencite[2]{meissner_2008}. Üblicherweise wird hier zwischen der sozial konstruierte Geschlechtsidentität oder Geschlechtsrolle (englisch \emph{gender}) und dem biologischen Geschlecht (englisch \emph{sex}) unterschieden \parencite[3f.]{meissner_2008}. Allerdings gibt es Menschen, welche sich keiner konstruierten binären Geschlechtisdentität zuordnen wollen oder können (oft als \emph{genderqueer}, \emph{genderfluid} oder \emph{non-binary} bezeichnet). Auch biologisch ist ein streng binäres Geschlechtermodell nicht haltbar -- intersexuelle Menschen weichen hier von einem solchen Modell ab und sind oft von Marginalisierung und Ausgrenzung betroffen \parencite[vgl.][]{richards_2016}. Um eine solche Marginalisierung zu vermeiden und auf die Vielzahl an Geschlechtern hinzuweisen, wird in dieser Arbeit eine genderinklusive Schreibweise genutzt. Die Schreibweise mittels des sogenannten \emph{Gendersternchen} (beispielsweise \enquote{Polikter*innen} soll hierbei zugleich Männer und Frauen, als auch alle weiteren Geschlechteridentitäten repräsentieren. Generell wird versucht, auf genderexklusive Begriffe (etwa \emph{Student}) zu verzichten und stattdessen eine genderinklusive Begriffe zu nutzen (\enquote{Studierende}). Da die überwiegende Forschung zur Repräsentanz von Geschlechtern in Parlamente ein binäres Geschlechtermodell anwendet -- etwa der Vergleich des Anzahls der Sitze von Frauen in repräsentativen Parlamenten durch die Inter-Parliamentary Union \textcite{ipu_2019} -- wird auch in dieser Arbeit vorwiegend ein binäres Modell verglichen. Möglichkeiten zur besseren Repräsentation von \emph{genderqueeren} und intersexuellen Menschen in Parlamenten finden sich beispielsweise bei \textcite{squires_2008}. Eine kritische Auseinandersetzung mit der hier beschriebenen Proplematik findet sich bei \parencite[6ff.]{meissner_2008}.} innerhalb der Parlamentsdebatten und das Nichtvorhandensein von geschlechterspezifischem Verhalten sowie eine geschlechterunspezifische Themenwahl. Da die Gleichstellung und Repräsentation von Frauen in Parlamenten anhand der Anzahl an Sitzen im Parlament in einer Vielzahl an Erhebungen bereits untersucht wurde\footnote{Eine Vergleich der Sitzanteile von Frauen nach Ländern findet sich etwa auf der Seite der Inter-Parliamentary Union unter \url{https://www.ipu.org/resources/publications/infographics/2019-03/women-in-politics-2019}}, wird in der hier vorliegenden Arbeit eine umfassendere Untersuchung der Repräsentation von Frauen vorgenommen. 

\citereset
\begin{quote}
     \enquote{Gender equal representation is not only about the proportion of men and women legislators or the outcomes of politics}\parencite[197]{erikson_2018}
 \end{quote}

In Anlehnung an \textcite{erikson_2018} wird eine geschlechtergerechte Repräsentation in der vorliegenden Arbeit weder daran gemessen, ob eine gleiche Anzahl an Frauen und Männern im Parlament vertreten ist, noch anhand der politischen Ergebnisse. Ausgangspunkt dieser Arbeit für eine \emph{geschlechtergerechtere Repräsentation} ist ein mehrdimensionaler Repräsentationsbegriff:

Repräsentation gilt als ein Kernkonzept in der Erforschung und Praxis der Politik und kann in verschiedenen, miteinander vernetzten Dimensionen betrachtet werden. Es kann hierbei zwischen dem, was repräsentiert wird, wie es repräsentiert wird und wer repräsentiert, unterschieden werden \parencite[557]{galligan_2007}. In den meisten Fällen bieten sowohl die Wähler*innen, Parteien, Wahlen und Gesetzgeber*innen den Hintergrund für eine Diskussion über eine oder mehrere Elemente der jeweiligen Repräsentationsdimension \parencite[557]{galligan_2007}.
In der vorliegenden Arbeit werden fünf Repräsentations-Dimensionen definiert und differenziert voneinander betrachtet. Obwohl keine der im Folgenden genannten Dimension gänzlich unabhängig von den anderen Dimensionen analysiert werden kann, werden sie in dieser Arbeit zunächst klar voneinander getrennt um die Komplexität der verschiedenen Dimensionen zu reduzieren. Eine Hierarchisierung der Dimensionen ist kein Bestandteil dieser Analyse. Das Ziel liegt in einer differenzierten Analyse der folgenden fünf Ebenen der Repräsentation von Frauen.

\begin{itemize}
    \item Ebene A: \emph{Repräsentation von Frauen im Parlament}: Sind Frauen im Parlament vertreten? Wie viele Frauen sind relativ und absolut im Parlament vertreten? 
    \item Ebene B: \emph{Beteiligung von weiblichen MPs an den Parlamentsdebatten}: Nehmen weibliche MPs absolut und relativ (in Bezug auf die Anzahl an Sitzen im Parlament) gleich viel an den Parlamentsdebatten teil wie männliche MPs? 
    \item Ebene C: \emph{Berücksichtigung von Frauen in der Sprache der Parlamentsdebatten}: Werden in den Debatten Frauen ebenso wie Männer sprachlich repräsentiert? Wird Gender-Fair-Language (GFL) genutzt? Wird GFL von weiblichen und männlichen MPs in gleichem Maße genutzt? 
    \item Ebene D: \emph{Vermeidung von geschlechtsbezogenen Stereotypen}: Werden gesellschaftliche geschlechtsbezogene Stereotype reproduziert? Werden unterschiedliche Themen von weiblichen und männlichen MPs behandelt? Gibt es thematische Unterschiede innerhalb der Reden zu den gleichen Themen zwischen männlichen und weiblichen MPs? 
    \item Ebene E: \emph{Verhinderung von frauenfeindlichen Verhaltensmustern}: Werden in den Parlamentsdebatten frauenfeindliche Verhaltensmuster reproduziert? Werden Frauen häufiger negativ unterbrochen? 
    
\end{itemize}

Der theoretische Hintergrund, die Relevanz des Untersuchungsgegenstandes sowie bisherige Untersuchungen zur Repräsentation von Frauen sollen zunächst detailliert dargestellt werden. Auf Basis dieses Hintergrunds ist eine anschließende Entwicklung von Hypothesen sowie die Konzeption des Forschungsdesigns möglich, welche eine umfassende Untersuchung der Repräsentation von Frauen in Parlamenten ermöglicht. Hierbei wird im besonderen auf die von \textcite[18]{back_2018} hingewiesenen Möglichkeiten der automatisierten Textanalyse durch neue empirische Methoden Gebrauch gemacht.

Im Anschluss soll die Forschungsfrage \enquote{\emph{Inwiefern unterscheiden sich Redebeiträge und Verhalten von weiblichen und männlichen Abgeordneten im 19. Deutschen Bundestag bezüglich Häufigkeit, Thematik und Geschlechterneutralität}} umfassend untersucht werden. Da der 19. Deutsche Bundestag umfassende Daten zu Reden und Verhalten der Abgeordneten zur Verfügung stellt (siehe Kapitel \ref{kapitel:datengrundlage}) und bereits in der Studie von \textcite{back_2018} thematisiert wurde und dort eine weitergehende Untersuchung empfohlen wird, bietet sich dies als Untersuchungsgegenstand an.

In der abschließenden Diskussion in Kapitel \ref{kapitel:diskussion} soll ausführlich auf die einzelnen Dimensionen der Repräsentation am Beispiel des 19. Bundestages eingegangen werden.


\Chapter{Theoretischer Hintergrund}

Nachfolgend soll insbesondere die Theorie des \emph{Workplace-Approach} von \textcite{erikson_2018} und das Konzept der \enquote{\emph{culture of masculinity}} von \textcite{lovenduski_2005} sowie \textcite{erikson_2018} dargestellt werden, welche maßgeblich für den Hintergrund und die weitere Argumentation dieser Arbeit dienen. 

\section{Workplace-Approach: Politik aus der Perspektive eines Arbeitsplatzes} 


Die bisherige Forschung zur politischen Repräsentation von Frauen fokussiert primär deskriptiven Themen, wie etwa die Frage danach, wie gesetzgebende Gremien durch Quoten geschlechtergerecht strukturiert werden können \parencites [vgl.]{dahlerup_2005}{schwindt-bayer_2009} oder substantive Themen, welche beispielsweise untersuchen, ob weibliche Gesetzgeberinnen (im Orignal: \emph{legislators}) einen positiven Einfluss auf genderfreundliche politische Ergebnisse haben \parencites [199]{erikson_2018}{beckwith_2007}. Den inneren Mechanismen der gesetzgebenden Körperschaften sowie der Frage, \enquote{how the political game itself is gendered} wurde bisher weniger empirische Aufmerksamkeit gewidmet \parencites[199]{erikson_2018}[vgl.][]{childs_2016}{dahlerup_2013}{wangnerud_2015}.

Ein Ausgangspunkt der vorliegenden Arbeit ist, dass das legislative Arbeitsumfeld für sich genommen ebenso wichtig ist, wie die Möglichkeiten der weiblichen Gesetzgeberinnen die Ergebnisse zu beeinflussen. Politikerinnen sollten in der Lage sein, ihre Aufgaben als Gesetzgeberinnen auf Augenhöhe mit ihren männlichen Kollegen zu erfüllen \parencite[199]{erikson_2018}. 

\textcites{dahlerup_2006}{dahlerup_1988} unterscheidet zwischen zwei verschiedenen Perspektiven der substantiellen Repräsentation von Frauen. Zum einen existiert die politische \emph{Outcome-Perspektive}, welche die wissenschaftliche Literatur zur substantiellen Repräsentation von Frauen tendenziell dominiert. Zum anderen existiert die Perspektive der Politik als \emph{Arbeitsplatz}, welche bisher weniger häufig diskutiert wurde \parencites[513]{dahlerup_2006}[199]{erikson_2018}. 

Während sich die Outcome-Perspektive darauf konzentriert, ob weibliche Gesetzgeberinnen den Inhalt politischer Entscheidungen beeinflussen, indem sie diese geschlechtsfreundlicher gestalten oder eine feministische Agenda verfolgen, beschäftigt sich die zweite Perspektive mit den Möglichkeiten, wie weibliche MPs als Repräsentantinnen gleichberechtigt mit ihren männlichen Kollegen auftreten können \parencites[199]{erikson_2018}{dahlerup_2006}{dahlerup_1988}.

Der in dieser Arbeit verwendete \emph{\enquote{workplace approach}} oder die bereits genannte Arbeitsperspektive zielt in Anlehnung an \textcite{erikson_2018} insgesamt darauf ab, die Repräsentation von Frauen aus einer breiteren Perspektive zu betrachten, indem der Fokus von den Ergebnissen auf die geschlechtsspezifischen Bedingungen innerhalb der Legislative verlagert wird. Mit Orientierung an die Arbeiten von \textcites{dahlerup_2006}{dahlerup_1988}{erikson_2018} werden die  Arbeitsbedingungen im deutschen Bundestag in der vorliegenden Arbeit als wichtig sowie als wesentlicher Bestandteil der Möglichkeiten, wie Frauen die politischen Ergebnisse beeinflussen können, eingestuft. Die sprachliche Einbeziehung von und das Verhalten gegenüber Frauen innerhalb der Parlamentsdebatten beeinflussen hierbei ebenso wie geschlechterneutrales Verhalten das Arbeitsumfeld von Poltikerinnen im Deutschen Bundestag.

Doch auch wenn sich in den niedergeschriebenen Regeln und den zugrundeliegenden Gesetzen des Bundestages keine \emph{per se} frauendiskriminierenden Passagen finden -- bis dato sind diese Insitutionen hauptsächlich durch Männer dominiert.

\todo{der letzte Satz ist etwas schwierig, ich kann noch nicht genau benennen was mich daran stört - daher kurz markiert um den nochmal im Auge zu behalten}

\section{Legislative Gremien als maskuline Organisation}

Das Geschlechterregime der legislativen Gremien wird, ebenso wie andere von Männern dominierten Sektoren, häufig als \enquote{\emph{permeated by a culture of masculinity}} beschrieben \parencites[200]{erikson_2018}{lovenduski_2005}.
In Anlehnung an \textcite{acker_1990} zeigt sich dies beispielsweise in der Existenz von formalen Regeln, die von Männern geschaffen wurden und einer männlich dominierten Organisation angepasst sind, sowie in Normen, wie sich ein (männlicher) Politiker präsentieren und verhalten soll \parencites[200]{erikson_2018}[48]{acker_1990}. Frauen werden infolgedessen mit dieser bereits existierenden \emph{culture of masculinity} konfrontiert, die als institutionelle Einschränkung fungieren kann wenn beispielsweise ihre Arbeit dadurch behindert wird \parencites[200]{erikson_2018}[47-56]{lovenduski_2005}. \textcite{erikson_2018} verweisen auf sozialpsychologische Erhebungen, wonach Frauen in männerdominierten Bereichen häufiger Diskriminierung ausgesetzt sind, da es an Übereinstimmungen zwischen den männerdominierten beruflichen Normen und den Eigenschaften mit denen Frauen typischerweise assoziiert werden, mangelt \parencites[vgl.][]{burgess_1999}{eagly_2002}{heilman_2001}{heilman_2004}. 
%
Frauen riskieren Diskriminierungen, weil sie entweder als weniger kompetent angesehen werden oder weil sie weibliche Attribute verletzen, wenn sie sich den Normen anpassen \parencite[200]{erikson_2018}. Die Folgen sind neben der Disqualifikation, die Abwertung der Leistungen sowie die ungleiche Behandlung von Frauen bis hin zu Belästigungen \parencites[200]{erikson_2018}{heilman_2001}{burgess_1999}:
%
\citereset
\begin{quote}
     \endquote{The gendered consequences of masculine norms are often manifested in the first case in the disqualification of women or the devaluation of their performance, whereas they often take the form of a disparate treatment of women in the second, including harassment} \parencite[200]{erikson_2018}
\end{quote}

Informelle Praktiken und Normen innerhalb des Arbeitsumfeldes können als Hindernisse für die Schaffung eines geschlechtergerechten Arbeitsumfeldes angesehen werden \parencite[200]{erikson_2018}. Ausgangspunkt der vorliegenden Arbeit ist die Annahme, dass der 19. Deutschen Bundestag ebenfalls von einer \enquote{\emph{culture of masculinity}} geprägt ist und u. a. dadurch die Geschlechtergerechtigkeit des Arbeitsumfeldes zum Nachteil von Frauen beeinflusst wird. 

\Chapter{Relevanz des Forschungsgegenstandes}

In demokratischen Systemen sind Parlamentarier*innen nicht nur wichtige Akteure als Gesetzgeber*innen, sondern auch als Teilnehmer*innen an der öffentlichen Debatte und als Mitwirkende an der öffentlichen Meinungsbildung \parencite[188]{dahlerup_2018}.
In Anlehnung an \textcite{back_2018} gibt es mehrere Gründe, warum die Auswirkungen des Geschlechts auf das Verhalten der legislativen Debatte interessant sind. Ein wichtiger Grund ist hierbei die zentrale Rolle der legislativen Debatte in parlamentarischen Demokratien, da Gesetze in der Regel von Abgeordneten diskutiert werden, bevor über sie abgestimmt wird \parencites[2]{back_2018}{back_2016}{proksch_2015}.
Die Debatten können in Anlehnung an \textcite[1]{proksch_2015} ebenso als Foren für öffentliche Kommunikation angesehen werden, welche von den Parteien und ihren Abgeordneten für Wahlzwecke genutzt werden \parencite[2]{back_2018}. Ausgehend davon, dass Debatten Auswirkungen auf die Politikgestaltung haben und von Parteien genutzt werden, um die Medien und die Wähler*innen zu informieren und zu beeinflussen, ist es von Relevanz ob dabei ein Geschlecht unterrepräsentiert ist \parencite[2]{back_2018}. \citeauthor{back_2018} argumentieren, dass eine solche Unterrepräsentanz im Gesetzgebungsprozess und der Medialen Berichterstattung zu einem Bias gegen Frauen führen kann und \enquote{gendered speech patterns} der Legitimität des demokratischen Systems schaden können \parencite[2]{back_2018}.
Die Relevanz von Geschlechtergleichheit auf politischer Ebene kann nicht (mehr) in Frage gestellt werden. Dennoch gibt es Uneinigkeit darüber, ob und inwiefern eine Geschlechtergleichheit bereits erreicht wurde und auf welchen Ebenen eine Gleichstellung der Geschlechter bereits abgeschlossen oder nahezu umgesetzt wurde. In Anbetracht der zunehmenden deskriptiven Repräsentation von Frauen in den Parlamenten ist es essentiell, eine Analyse nicht ausschließlich auf dieser Ebene der Repräsentation durchzuführen und sich optimistisch mit einer Zunahme an Frauen in den Parlamenten zufrieden zu geben, sondern die Forderung nach Geschlechtergerechtigkeit, auch im Sinne gleicher und gleichwertiger Repräsentation auf allen Ebenen zu fordern und fördern. 

\citereset
\begin{quote}
    \enquote{While women have made substantive progress in their representation in politics, they are still well underrepresented in political life in most nations} \parencite[2]{coffe_2013}
\end{quote}

Wenngleich ein politischer Konsens darin besteht, dass Gleichheit und Gerechtigkeit, sowie implizit Geschlechtergleichheit und ,-gerechtigkeit eine fundamentale Komponente von Demokratie darstellt, gibt es weltweit noch immer weniger Frauen (24,3\% \parencite[]{ipu_2019} in den Parlamenten (ausgenommen sind hierbei derzeit Ruanda: 61,3\%, Cuba: 53,2\% und Bolivien: 53,1\% - Stand April 2019 \parencite{ipu_2019}).  

Anne Phillips gilt als eine der wichtigsten Vertreterinnen auf dem Gebiet der parlamentarischen Repräsentativität von Frauen -- sie baut in ihrem Werk \emph{The Politics of Presence} \parencite*{phillips_1998} die Annahme auf, dass eine höhere deskriptive Repräsentanz durch Sitze im Parlament auch zu einer höheren substanziellen Repräsentanz von Frauen \parencite[52]{wangnerud_2009}. Hierfür analyisiert Phillips unter anderem Umstände, unter denen Frauen als Kandidatinnen für die Parlamentssitze ausgewählt werden, erfolgreich an Wahlen teilnehmen und zu Ministerinnen werden \parencite[vgl.][416f.]{blaxill_2016}. Frauen sind im täglichen Leben anderen Erfahrungen ausgesetzt als Männer. Hierbei bezieht sich \textcite{phillips_1998} unter anderem auf die Kindererziehung, Bildung, Auswahl an Berufen, Unterscheidung von bezahlter und unbezahlter Arbeit sowie auf Gewalterfahrungen und sexuellen Belästigungen. \textcite{phillips_1998} plädiert aufgrund der unterschiedlichen Lebenserfahrungen für die notwendige Repräsentation von Frauen im Parlament, um andere Frauen vertreten zu können \parencite[vgl.][52]{wangnerud_2009}.

\citereset
\begin{quote}
    \enquote{There are particular needs, interests, and concerns that arise from women's experience, and these will be inadequately addressed in a politics that is dominated by men. Equal rights to a vote have not proved strong enough to deal with this problem; there must also be equality among those elected to once} \parencite[66]{phillips_1998}
\end{quote}

Hanna F. Piktin \parencite*{pitkin_1972} argumentierte bereit 1967 in ihrem Hauptwerk \emph{The Concept of Representation}, dass der Interessenbegriff in der Repräsentationsdebatte omnipräsent (\enquote{ubiquitious}) ist \parencite[69]{wangnerud_2000}. Für Pitkin sind es die Parlamentarier\emph{innen}, welche sich den Wünschen und Interessen, dem Wohlergehen sowie Themen der Frauen widmen und diese vertreten \parencites[vgl.][413]{blaxill_2016}{pitkin_1972}. In Anlehnung an Phillips' Theorie der \emph{politics of presence} \parencite*{phillips_1998} wird angenommen, dass weibliche Politikerinnen die Interessen und Wünsche weiblicher Bürgerinnen besser als männliche Politiker repräsentieren können. Hierbei wird die deskriptive mit der substantiellen Repräsentation verknüpft \parencite[52]{wangnerud_2009}: ein höherer Frauenanteil in Parlamenten führt zu einer stärkeren Thematisierung der Belange von Frauen.

Während die Forderungen einer deskriptiven parlamentarischen Geschlechtergleichheit ebenso wie die Forderungen nach substantieller Geschlechtergleichheit bereits ihre Wege in die wissenschaftliche Diskussion und in die Politik gefunden haben, wurde der geschlechtergerechten Sprache sowie dem geschlechtsbezogenen Verhalten in den Parlamenten bisher weniger Aufmerksamkeit gewidmet. 
Laut \textcite{menegatti_2017} ist die Sprache jedoch eine der einflussreichsten Faktoren, wodurch Sexismus und Geschlechterdiskriminierung gefördert und reproduziert werden \parencite*[1]{menegatti_2017}. Sprache kann hierbei insbesondere die sozialen Asymmetrien von Status und Macht zugunsten des Mannes reproduzieren \parencite[1]{menegatti_2017}. Es existier[t]en einvernehmliche Normen, wonach der prototypische Mensch ein Mann ist, was sich in den Strukturen vieler Sprachen widerspiegelt und darin verankert ist. Viele grammatikalische und syntaktische Regeln sind so aufgebaut, dass weibliche Ausdrücke normalerweise von entsprechenden männlichen Formen abgeleitet werden \parencite*[1]{menegatti_2017}. Männliche Substantive und Pronomen werden beispielsweise häufig mit einer generischen Funktion verwendet um sich sowohl auf Männer als auch auf Frauen zu beziehen. Auf diese Weise verschwinden Frauen aus der mentalen Repräsentationen \parencites{vaughan_2018}{stahlberg_2001}. Maskuline Generika lassen Leser*innen und Hörer*innen mehr in männlichen als weiblichen Personenkategorien denken \parencites[2]{sczesny_2016}{stahlberg_2007}.

\citereset
\begin{quote}
    \enquote{Given that language not only reflects stereotypical beliefs but also affects recipients’ cognition and behavior, the use of expressions consistent with gender stereotypes contributes to transmit and reinforce such belief system and can produce actual discrimination against women} \parencite[2]{menegatti_2017}
\end{quote}

\todo[inline]{gendered -shelley et al ? (ich konnte dem Text nicht sehr viel entnehmen!)
Fehlt hier noch - Stereotypen als Theoretischer Unterbau und Unterbrechungen ...JOSEF: Die Stereotype würde ich dann beim Forschungsstand aufführen.}

Neben einer mangelnden deskriptiven Repräsentanz in den meisten repräsentativen Institutionen parlamentarischer Demokratien \parencite[vgl.][]{ipu_2019} ist also auch eine mangelnde Repräsentanz von Frauen und anderen, marginalisierten Gruppen bei vielen Sprachen festzustellen. Dies kann sich doppelt negativ auf die substanzielle Repräsentanz von Frauen in parlamentarischen Demokratien äußern -- sie sind weder durch Sprache noch durch Anwesenheit \emph{sichtbar}. Ein solch mangelnder Einfluss auf den legislativen Prozess kann sich wiederum durch Gesetze äußern, welche sich gegen die Interessen von Frauen richten \parencite[2]{back_2018}.


\section {Derzeitiger Forschungsstand und relevante Studien}

\todo[inline]{Auch hier fände ich wieder einen einleitenden Absatz gut. Ich überleg mir gleich noch was.}

\subsection{Frauen im Parlament}

In der Forschung zu Frauen im Parlament wird generell zwischen einer deskriptiven, substantiellen und (in einigen Fällen) der symbolische Form der Repräsentation unterschieden. Bei der substantiellen Repräsentation von Frauen handelt es sich um ein bisher weniger wissenschaftlich erforschtes Feld als die Analyse der deskriptiven Repräsentation \parencite[59]{wangnerud_2009}. 
Letzteres legt den Schwerpunkt auf die Analyse der Anzahl von Frauen in repräsentativen Institutionen, ersteres untersucht hingegen die Auswirkung der Präsenz von Frauen in Parlamenten \parencites[14]{coffe_2013}[52]{wangnerud_2009}.
Die zentralen Fragen lauten hierbei:

\citereset
\begin{quote}
  \enquote{[W]hether the widely professed aspiration to feminise democracy -- and in so doing to politically empower women -- is a matter largely of symbolism or substance […]}
  \enquote{[…] wheter the priority should simply be to increase the proportion of women MPs in Parliament […] or as Hanna Pitkin argued in 1967, to represent minds as well as bodies.}
  \parencite[413]{blaxill_2016}
\end{quote}

%
Die Theorie der \emph{politics of presence} \parencite{phillips_1998} diente \textcite{wangnerud_2000} als Ausgangspunkt für ihre Forschung im Schwedischen Parlament (\emph{Testing the Politics of Presence, Women's Representation in the Swedish Riksdag}, 2000). Wägnerud prüft hierbei die Hypothese, ob weibliche Politikerinnen die Interessen von Frauen stärker vertreten als männliche Politiker \parencite[84]{wangnerud_2000} und kommt zu folgendem Ergebnis: \enquote{[It is] difficult to repudiate the conclusion that women's interest are primarily represented by female politicians} \parencite[][84]{wangnerud_2000}. Laut\citeauthor{wangnerud_2000} bleibt es jedoch ausstehend welche Auswirkungen zu erwarten sind, wenn die Zahl der Frauen im Parlament steigt \parencite{wangnerud_2009}.
Die Forschung von \textcite{celis_2008} greift diese Annahme empirisch auf und untersucht, in welchem Ausmaß die Bedürfnisse und Interessen von Frauen gesteigert werden, wenn die Anzahl von Frauen in politischen Entscheidungen zunimmt \parencite[vgl. auch][4]{galligan_2016}. Ein einfacher Anstieg der Anzahl von Frauen reicht nach \textcite{celis_2008} nicht aus, um einen signifikanten Einfluss auf den Gesetzgebungsprozess zu erreichen. In Anlehnung an \textcite{caul_2001} sei es beispielsweise notwendig, dass weibliche MPs in einflussreichen, wichtigen Positionen vertreten sind, um substantielle Repräsentation zu ermöglichen \parencites{caul_2001}[vgl. auch][14]{coffe_2013}.
Auch die Forschung von \textcite{back_2018} die Auswirkungen, wenn die Zahl der Frauen im Parlament steigt und kommt zu folgendem Ergebnis: 

\citereset
\begin{quote}
  \enquote{The variation found here is, however, not in line with the hypothesis suggesting that increased descriptive representation should lead to a rise in the substantive representation so that the prediction drawn from the critical mass theory is not given support here. Instead, we find that women are more underrepresented in legislative debates when they represent parties with many female MPs.}
  \parencite[17]{back_2018}
\end{quote}

\textcite{back_2018} untersuchen hierfür länderübergreifend sieben europäische Länder bezüglich der Themen und Redeanteile weiblicher MPs und resümieren, dass Frauen in Parlamenten seltener das Wort ergreifen. Dieses Ergebnis ist nicht auf einen generell niedrigeren Anteil an Frauen in den Parlamenten zurückzuführen, so \textcite{back_2018}. Frauen in Parteien mit einem geringen Frauenanteil ergreifen sogar häufiger das Wort als Frauen in Fraktionen mit einem hohen Anteil an weiblichen Mitgliedern \parencite*[17]{back_2018}. Die \emph{critical mass theory} konnte somit in der Arbeit von \textcite{back_2018} nicht nachgewiesen und sogar ein gegenteiliger Effekt beobachtet werden. Allerdings betonen die Autor*innen in ihrer Untersuchung auch, dass sich dieser Effekt bei den untersuchten Parteien und Parlamenten erheblich unterscheiden \parencite[17]{back_2018}. Eine Beleuchtung der Hintergründe eines solchen \enquote{backlash effect} -- Frauen sprechen seltener in Parlamenten bei einer größeren Präsenz von Frauen im Parlament -- und Kritik an der \emph{critical mass theory} findet sich zusammenfassend bei \textcite[4]{back_2018} und tiefgreifender bei \textcite{childs_2008}.
\todo[inline]{Hier war vorher der Hinweis, ob wir die critical mass theory aufmachen sollten -- ich hab in dem vorherigen Absatz das mal zusammengefasst, aber vielleicht sollte man da noch einen kleinen Absatz bei der Politics of Presence dazu schreiben. Reicht ja wirklich zwei Sachen.
Die \emph{critical mass theory} von XXX geht davon aus, dass die Interessen von Personengruppen (in diesem Fall Frauen) erst Wahrnehmung finden, wenn diese in einer \enquote{kritischen Masse} im legislativen Prozess beteiligt ist. Erst wenn mehr als nur ein paar wenige Frauen in einem Parlament vertreten sind, ist es ihnen möglich, effektiv zusammenzuarbeiten und \enquote{frauenfreundliche} Inhalte im legislativen Prozess durchzusetzen \parencite[752]{childs_2008}.

Könnte man einfach nach Philips Politics of Presence setzen.}

Aus den Untersuchungen von \textcite{kathlene_1994} zu US-Staatsgesetzen und \textcite{back_2014} zum schwedischen Riksdag geht hervor, dass Frauen in Debatten seltener sprechen, während die Fallstudien von \textcite{broughton_1999}, \textcite{murray_2010} und \textcite{wang_2014} dies nicht beobachten konnten. In Anlehnung an \textcite{back_2018} kann davon ausgegangen werden, dass es eine länderübergreifende Variation der Rolle des Geschlechts in Debatten gibt \parencite[2]{back_2018}. Eine gesonderte Untersuchung der Rolle des Geschlechts in den Debatten des Deutschen Bundestags erscheint auch aus dieser Perspektive notwendig. 
\todo[inline]{[Das könnte im weiteren verlauf vll eine gute Verbindung sein, aber noch zu prüfen::!!! Taylor-Robinson (2014) looks at debate participation as one type of substantive representation and refers to a seminal study of US state legislatures which shows that women speak less frequently in debates \parencite{kathlene_1994}.}
\todo[inline]{Irgendwas fehlt da aber noch, oder?}

In einer früheren Studie konnte zudem festgestellt werden, dass geschlechterbezogene Unterschiede bei der gewählten Themen der Reden festzustellen sind \parencite[514f.]{back_2014}. \textcite{back_2014} konnten in ihrer Untersuchung des schwedischen Parlaments nachweisen, dass Männer im schwedischen \emph{Riksdag} häufiger zu sogenanntenn \emph{hard policies} (etwa Wirtschaft, Energie, Infrastruktur) sprechen. Bei sogenannten \emph{soft policies} (Erziehung, Soiale Sicherung, Bildung) war hingegen kein Unterschied bezüglich des Redeanteils und Geschlecht der Abgeordneten festzustellen \parencite[514f.]{back_2014}. 

Eine umfassende Untersuchung von Geschlechterunterschieden in Bezug auf Sprache, Thematik sowie Verhalten der Parlamentsmitglieder des Deutschen Bundestages ist allerdings bis zu diesem Zeitpunkt noch ausstehend, wenngleich erforderlich: 

\citereset
\begin{quote}
 \enquote{Our research clearly suggest that gender plays a role in parliamentary speech-making and the selection of the MPs who take the parliamentary floor, which calls for further comparative research on the role of gender in legislative  debates in different institutional contexts and with varying degrees of descriptive representation}
  \parencite[515]{back_2014}
\end{quote}

\todo[inline]{\citeauthor{back_2014} haben in ihrer Studie nur untersucht, welcher Minister vor einer Rede spricht und danach die Thematisierung festgestellt. In der neueren, länderübergreifenden Untersuchung machen \citeauthor{back_2018} darauf aufmerksam, dass man die Thematisierung in den Reden durch neue Möglichkeiten der automatisierten Inhaltsanalyse wie beispielsweise der \emph{Latent Semnatic Analysis} bei \textcite{gustafssonsenden_2015} auf Parlamentsdebatten anwenden sollte, um diese Thematik tiefergehend zu erforschen-}

\subsection{Geschlechterspezifische Unterschiede im Parlament}\label{kapitel:geschlechterunterschiede}

In Bezug auf die Interaktion zwischen Gesetzgeber*innen konnten frühere Untersuchungen zeigen, dass Frauen und Männer innerhalb der gesetzgebenden Gremien nicht nur unterschiedliche Politik betreiben, sondern außerdem unterschiedlich behandelt werden \parencites[201]{erikson_2018}{childs_2004}. Zudem gibt es Hinweise darauf, dass männliche MPs bei einem zunehmenden Frauenanteil im Parlament, unhöflich und respektlos gegenüber weiblichen Gesetzgeberinnen sein können \parencites[201]{erikson_2018}{kathlene_1994} und dass weibliche Gesetzgeberinnen offener Diskriminierungen und sexueller Belästigung ausgesetzt sind \parencites[201]{erikson_2018}[76]{lovenduski_2005}{lovenduski_2004}.

Die Untersuchungen von \textcite{erikson_2018} liefern neue Erkenntnisse über die Existenz von anhaltenden geschlechtsspezifischen Mustern im schwedischen Parlament. Wenngleich das schwedische Parlament im Bereich der Gleichstellung der Geschlechter formal und deskriptiv vielen anderen Ländern voraus ist, lassen die Ergebnisse dennoch auf Elemente einer maskulinen Kultur hinweisen \parencite[211]{erikson_2018}:

\citereset
\begin{quote}
    \enquote{Although changing formal rules may end officially sanctioned gender discrimination, more than two decades of nearly genderequal descriptive representation have not been enough to change informal norms and practices and overcome all institutionalized forms of male bias} \parencite[211]{erikson_2018}
\end{quote}

 Dies ist unabhängig davon, dass es sich beim schwedischen Parlament formal um ein geschlechtergerechten Arbeitsplatz handelt und sich die Regeln und Vorschriften für Männer und Frauen nicht unterscheiden \parencite[211]{erikson_2018}. In Anlehnung an \citeauthor{erikson_2018} stellt sich hierbei die Frage, aus welchem Grund weibliche MPs mehr Druck, Angst und negative Behandlungen erfahren. Es wird vermutet, dass informelle Aspekte des Arbeitsumfelds im Parlament in einer Weise geschlechtsspezifisch sind, die weibliche MPs trotz ihrer formalen und deskriptiven Geschlechtergleichstellung weiterhin benachteiligen \parencite[210]{erikson_2018}.

\citeauthor{erikson_2018} verweisen auf die Notwendigkeit einer Gleichstellungsarbeit im Parlament, welche nicht ausschließlich die formalen Aspekte der Arbeitsbedingungen berücksichtigen, sondern ebenso das Bewusstsein der MPs für die Notwendigkeit der Transformation etablierter Normen und Praktiken schärfen \parencite[211]{erikson_2018}.
Mit der vorliegenden Arbeit wird teilweise auf die Forderung von\citeauthor{erikson_2018} eingegangen, die von Ihnen festgestellten Ergebnisse mit den empirischen Studien anderer legislativen Körperschaften und unterschiedlichen formalen Bedingungen (in diesem Fall der 19. Deutsche Bundestag) zu vergleichen um jene Mechanismen zu identifizieren, welche geschlechtsspezifische Arbeitsbedingungen für die Gesetzgeber*innen fördern oder behindern \parencite[211]{erikson_2018}. Letzeres kann in dieser Arbeit allerdings nicht geleistet werden und bedarf weiterer Forschung. 


\subsection{Gender-Fair-Language zur Reduktion von Stereotypisierungen und Reduktion von Diskriminierung}\label{kapitel:gfl-studien}

\todo[inline]{Aktueller Forschungsstand . ggf. sczency formanowicz und moser}
\todo[inline]{Wir sollten hier unbedingt noch Formanowicz reinnehmen - einfach auch um Besipiele aufzuzeigen, was eigentlic GFL ist weil wir das später noch verwenden -- also sowas wie \enquote{he or she} weil wir später noch von \enquote{Kolleginnen und Kollegen} schreiben. Kann ich aber selbst noch ergänzen wenn gewünscht, !!!!GG: GERNE!!!}


Nach \textcite{menegatti_2017} handelt es sich bei der Sprache um eine der einflussreichsten Faktoren, wodurch Sexismus und Geschlechterdiskriminierung gefördert und reproduziert werden \parencite*[1]{menegatti_2017}. Es existier[t]en einvernehmliche Normen, wonach der prototypische Mensch ein Mann ist, was sich in den Strukturen vieler Sprachen widerspiegelt und darin verankert ist. Sprache kann soziale Asymmetrien von Status und Macht zugunsten des Mannes reproduzieren \parencite{menegatti_2017}.
\todo[inline]{Problematisch: Exakt den selben Absatz haben wir schon wieter oben aufgeführt -- entsprechend können wir in hier nicht einfach noch mal nutzen. Der muss raus und anders eingeleitet werden. Hier muss mehr über die Studien geredet werden, Menegatti ist mehr Theorie.}
Maskuline Generika werden geschlechterspezifisch in Bezug auf männliche Personen genutzt sowie generell in Bezug auf Gruppen mit gemischten Geschlechtern und in Bezug auf Personen oder Gruppen deren Geschlecht ungewiss oder irrelevant ist, mit folgendem Ergebnis: \enquote{In this way they equate maleness and humanness} \parencite[169]{stahlberg_2007}. Durch die maskulinen Generika verschwinden Frauen aus der mentalen Repräsentationen \parencites{vaughan_2018}{stahlberg_2001}.In der feministischen Linguistik wird generell angenommen, dass maskuline Bezeichnungen weibliche Personen weniger vorstellbar und sichtbar machen \parencite[131]{stahlberg_2001}. Maskuline Generika lassen Leser*innen und Hörer*innen mehr in männlichen als weiblichen Personenkategorien denken \parencites[2]{sczesny_2016}{stahlberg_2007}. In Anlehnung an \textcite{stahlberg_2007} wird in der feministischen Sprachkritik davon ausgegangen, dass die maskulinen Generika aus einem grundlegenden Androzentrismus entspringen (\enquote{\emph{the Male-as-Norm-view}}), diesen aufrechterhalten und die Frauen unsichtbar machen \parencites[170]{stahlberg_2007}{miller_1976}{silveira_1980}.

\citereset
\begin{quote}
    \enquote{Given that language not only reflects stereotypical beliefs but also affects recipients’ cognition and behavior, the use of expressions consistent with gender stereotypes contributes to transmit and reinforce such belief system and can produce actual discrimination against women} \parencite[2]{menegatti_2017}
\end{quote}

Die Verwendung von \emph{gender-fair linguistic} (GFL, auch als \emph{gender-fair language} bezeichnet) kann diese negativen Auswirkungen effektiv verhindern und Geschlechtergerechtigkeit fördern \parencite[1]{menegatti_2017}.
GFL wurde unter anderem im Rahmen eines umfassenden Versuchs zur Verringerung von Stereotypen und Diskriminierung in der Sprache eingeführt \parencite[2]{sczesny_2016}. \todo[inline]{Hier sollten wir mehr auf die Ergebnisse eingehen, nicht was draus geschlussfolgert wurde.} Feminiserung und Neutralisierung sowie die Kombination beider sind die bevorzugten Strategien einer geschlechtergerechten oder geschlechterneutralen Sprache. Auch die Verwendung von Wortpaaren (Lehrerinnen und Lehrer) zählt ebenso zu den Ausdrucksformen der GFL. Neben Wortpaaren sind auch geschlechtsneutrale Formen (Studierende statt Student) mögliche GFL-Ausdrucksformen \parencite[2]{sczesny_2016}.


Aufgrund der oben genannten Benachteiligungen für die Frauen ist es notwendig, die Sprachgewohnheit dahingehend zu ändern, GFL umfassend zu etablieren, um Vorurteile bezüglich der Geschlechter zu reduzieren und gleichzeitig eine Reproduktion von Stereotypen zu vermeiden. Die Verwendung von geschlechtergerechten Ausdrücken anstelle von maskulinen Generika ist für den Abbau von Geschlechtervoreingenommenheit und die Förderung der Gleichstellung der Geschlechter laut Menegatti und Rubini unabdingbar \parencite*{menegatti_2017}.

Die Umsetzung und Etablierung von GFL hat in verschiedenen Ländern bisher unterschiedliche Stadien erreicht und wird beispielsweise von der UNESCO und der Europäischen Kommission empfohlen und in deren Dokumenten angewandt \parencite[4]{sczesny_2016}.

\citereset
\begin{quote}
    \enquote{[…] language does not merely reflect the way we think: it also shapes our thinking. If words and expressions that imply that women are inferior to men are constantly used, that assumption of inferiority tends to become part of our mindset; hence the need to adjust our language when our ideas evolve.} \parencite {unesco_2011}
\end{quote}

Bezugnehmend auf die Förderung des Gebrauchs von GFL verweisen \textcite{koeser_2014} mit Ihrer Studie auf die Notwednigkeit einer Auseinandersetzung mit den Argumenten, für die Nutzung von GFL:
\begin{quote}
    \enquote {The present study is the first one that investigated persuasion by arguments concerning gender-fair language. It indicates that arguments can be an effective tool for making speakers use gender-fair language.} \parencite[556]{koeser_2014}
\end{quote} 

\todo[inline]{Dieser Teil müsste noch umgebaut werden. Er geht viel zu sehr auf den theoretischen Background und nicht auf die Ergebnisse relevanter Studien zu dem Thema ein..}

\Chapter{Übergeordete Forschungsfrage und Konzeption der Arbeit}

Während sich, wie eingangs beschrieben, eine Vielzahl der wissenschaftlichen Forschung mit der Repräsentation von Frauen in Parlamenten aus einer deskriptiven und substantiellen Perspektive auseinandersetzt, fokussiert die vorliegende Forschungsarbeit die geschlechterspezifischen Repräsentationsunterschiede bezüglich der Parlamentsdebatten auf fünf Ebenen (siehe Kapitel \ref{Einleitung}). 

\begin{itemize}
    \item Ebene A: Repräsentation von Frauen im Parlament
    \item Ebene B: Beteiligung von weiblichen MPs an den Parlamentsdebatten
    \item Ebene C: Berücksichtigung von Frauen in der Sprache der Parlamentsdebatten
    \item Ebene D: Vermeidung von geschlechtsbezogenen Stereotypen
    \item Ebene E: Verhinderung von frauenfeindlichen Verhaltensmustern
\end{itemize}

Neben dem Anteil an Reden von Frauen im Deutschen Bundestag werden die Reden des Deutschen Bundestages inhaltlich analysiert und das Verhalten der Abgeordneten in Bezug auf geschlechtsspezifische Besonderheiten untersucht. Die zentrale Forschungsfrage lautet: \enquote{\emph{Inwiefern unterscheiden sich die Redebeiträge und das Verhalten von weiblichen und männlichen Abgeordneten im 19. Deutschen Bundestag bezüglich Häufigkeit, Thematik und Geschlechterneutralität}}.


\section{Hypothese Genderinklusive Sprache}\label{kapitel:hypothese1}

Auf Basis der Argumentation von Wängnerud \parencites*{wangnerud_2000}{wangnerud_2009} wird erwartet, dass sich vor allem Frauen über die Ausgrenzungserfahrung durch Sprache bewusst sind, entsprechende Erfahrungen bereits gemacht haben und daher eher eine geschlechterneutrale und geschlechtergerechte Sprache verwenden als Männer. 
\todo[inline]{WOHER IST DAS? ARGUMENTATION VON WÄNGNERUD FINDE ICH NICHT!!!???!!!???!}
\todo[inline]{Wängnerud ist der Überzeugung, dass Frauen andere Erfahrungen gemacht haben als Männer, auch Ausgrenzungserfahrung -- aber eben auch Erziehung usw. Deshalb meint sie doch, dass es wichtig ist, dass Frauen in Parlamenten sind, ansonsten würden diese Erfahrungen gar nicht umgesetzt werden -- orientiert sich auch ziemlich an Phillips. Da Frauen diese Erfahrungen vermutlich machen mussten, werden sie wahrscheinlich eher das nutzen. Deswegen ist es auch nur die Basis der Argumentation Wängneruds, und nicht ihre Aussage.}

\textcite{koeser_2014} untersuchten, inwiefern Personen ermutigt werden können, eine geschlechtergerechte Sprache zu verwenden und zu akzeptieren. \todo[inline]{(WELCHE Sprecherinnen, wo!!!???)} \todo[inline]{Da ging es um die generelle Nutzung von GFL und wie man sie promotet, sollte aber in die relevaten Studien hoch, da ist GFL sowieso noch nicht so gut abgedeckt.} Die Ergebnisse zeigten, dass Personen ihr Sprachverhalten zu einem geschlechtergerechten Gebrauch hin änderten, wenn sie entsprechenden Argumenten ausgesetzt waren. Die Studie bestätigt, dass Argumente zur Förderung einer geschlechtergerechten Sprache motivieren können, geschlechtergerechte Formulierungen zu verwenden \parencite[548]{koeser_2014}. Auffällig war allerdings, dass Frauen ihren Sprachgebrauch eher als Männer in Bezug auf geschlechtergerechte Sprache ändern. Eine Erklärung hierfür vermuten die Autorinnen in den Korrelationsmustern: Der zuvor gemessene Sprachgebrauch korrelierte signifikant mit den Einstellungen, wobei Frauen eine geschlechtergerechte Sprache mehr bevorzugten als Männer\parencite[555]{koeser_2014}.
\todo[inline]{Der ganze Absatz und das Zitat müssen in das Kapitel relevante Studie zu GFL.}

\citereset
\begin{quote}
    \enquote {Surprisingly, women changed their language use more in the direction of gender-fair language than men.} \parencite[555]{koeser_2014}
\end{quote}

\todo[inline]{Das Thema GFL ist leider noch das dürftigste mMn. Alle anderen sind super. Mehr Literatur schadet uns da auf keinen Fall.}

Ausgehend von diesen Ergebnissen, stellen wir die folgende Hypothese auf: 

\textbf{Hypothese 1:} \emph{Frauen verwenden in ihren Reden häufiger gender-fair language als Männer}


\section{Hypothese Themen der Reden}\label{kapitel:hypothese2}

\begin{quote}
    \enquote{Legislatures thus remain strongly gendered institutions, with a marked division of labor between men and women politicians that becomes especially visible in the realm of social policy ” \parencite[250]{ennser-jedenastik_2017}}
\end{quote}

Es gibt diverse Erklärungsansätze für die unterschiedliche Themenwahl von Männern und Frauen bei Parlamentsdebatten. Laut \textcite{wangnerud_1996}  gibt es zwei mögliche Erklärungsansätze für eine \emph{horizontale Arbeitsteilung} bezüglich der Zuordnung von MPs zu den jeweiligen Gremien. Zum einen nennt \citeauthor{wangnerud_1996} hierbei die Stereotypisierungen dessen, was als männlich oder weiblich angesehen wird. Zum anderen kann es auch Unterschiede bezüglich der Interessen geben \parencite{wangnerud_1996}. \todo[inline]{HIER WICHTIG SPÄTER KRITISCH REAGIEREN!!!!} \todo[inline]{Hier sollten wir auch genauer zitieren, wo findet sich diese Argumentation bei Wägnerud?} \citeauthor{back_2014} verweist diesbezüglich darauf, dass die Präferenzen und Entscheidungen der MPs nicht unabhängig von der Zugehörigkeit zu einer sozialen Gruppe und von den eigenen Lebenserfahrungen analysiert werden können. Die MPs beziehen sich bei der Entwicklung ihrer Präferenzen bewusst oder unbewusst auf ihre Biographie, so \textcite[507]{back_2014}. 
In den Jahren 1985, 1988 und 1994 wurden Umfragen der schwedischen MPs durchgeführt, anhand derer \textcite[81]{wangnerud_2000} feststellen konnte, dass es einen Zusammenhang zwischen dem Geschlecht der Politiker*innen und dem Ausmaß, in dem sie sozialpolitische Fragen verfolgen, gibt \parencites[506]{back_2014}[82]{wangnerud_2000}. Weibliche MPs sprechen in ihrem Wahlkampf vermehrt zu Themen der Sozialpolitik, der Familienpolitik, der Seniorenpflege sowie des Gesundheitswesens. Zudem wurde von weiblichen MPs betont, dass Wohlfahrtsfragen in ihren eigenen Interessengebieten liegen. Laut \textcite[82]{wangnerud_2000} waren Politikerinnen immer die Gruppe, die sozialpolitische Themen in ihrer parlamentarischen Arbeit am stärksten verfolgt haben \parencites[vgl.][507]{back_2014}.
In ihrer Untersuchung zum schwedischen Parlament konnten \textcite{back_2014} nachweisen, dass Geschlechterunterschiede bezüglich der Themen der Reden festzustellen sind. Männer sprechen laut \textcite{back_2014} im schwedischen \emph{Riksdag} häufiger zu \emph{hard policies} (Wirtschaft, Energie, Infrastruktur). Bei \emph{soft policies} (Erziehung, Soziale Sicherung, Bildung) ist hingegen kein Unterschied bezüglich des Redeanteils und dem Geschlecht der Abgeordneten festzustellen \parencite[514f.]{back_2014}. Eine Analyse der Reden bezüglich möglicher Themenunterschiede zwischen den Geschlechtern, auf Basis der bisherigen Befunde im schwedischen Parlament (\parencites{wangnerud_2000}{wangnerud_2009}{back_2014}), ist für den 19. Deutschen Bundestag bisher noch ausstehend. Daher wird im Folgenden untersucht, ob eine geschlechtsspezifische, unterschiedliche Thematisierung in den Reden feststellbar ist: 
\todo[inline]{!!!!(NOCH MEHR BEGRÜNDEN; WARUM DAS RELEVANT IST - STEREOTYPEN FÖRDERN,etc...)!!!!}
\todo[inline]{Teilweise haben wiir hier wieder aussagen doppelt. Ich würde sagen, mehr nach oben und in den Hypthesen eher verweisen auf die relevante Literatur.}

\textbf{Hypothese 2:} \emph{Es sind unterschiedliche Thematisierungen in den Reden der Abgeordneten feststellbar.}

Wie \textcite{back_2014} bereits aufzeigen konnte, gibt es unterschiedliche Erklärungsansätze. Eine ausführliche Auseinandersetzung mit den Ursachen der möglichen Themenunterschiede kann in der vorliegen Arbeit nicht geleistet werden. Mögliche Gründe für Themenunterschiede können (a) Entscheidungen der MPs aufgrund ihrer persönlichen Interessen, (b) Stereotype und/oder Normen in der Gesellschaft und/oder innerhalb der Parteien oder (c) Strategien, Hierarchien oder bestimmte Netzwerke innerhalb der Parteien sein \parencites[507]{back_2014}. \textcite [250]{ennser-jedenastik_2017} betont bezugnehmend auf \textcites{baekgaard_2012}{thomas_1994} dass es sich bei der Wahl von sozialpolitischen Themen der Frauen um die Selbstauswahl und nicht die offene Diskriminierung dieses Musters handelt \parencite[250]{ennser-jedenastik_2017}. \textcite{herrnson_2003} berichtet, dass weibliche Kandidatinnen, die stereotypische weibliche Themen betonen und in ihren Kampangnen Frauengruppe ansprachen, durch diese Strategie einen Wahlvorteil erhielten \parencite[250]{ennser-jedenastik_2017}. Auch \textcite{celis_2018} nehmen Bezug auf die unterschiedliche Themenwahl von männlichen und weiblichen MPs und begründen diese auf potentielle Erfolgsaussichten:

\citereset
\begin{quote}
    \enquote{Women’s issues and interests can be represented with greater success when they fit the dominant views and discourses about the type of issues and interests that representatives should be concerned with. However, these have a gendered and elitist bias, and strategically framing decisions [...] to fit with these dominant conceptions, by definition, distorts the selection of the types of issues and interests that will be represented” \parencite[151]{celis_2018}}.
\end{quote}

\textcite{celis_2018} gehen in Ihrer Argumentation soweit, dass Themen und Interessen, die im Widerspruch zu einer dominanten Auffassungen stehen, strategisch und/oder unbewusst durch Themen ersetzt werden, die akzeptabler sind und eine bessere Chance auf Unterstützung haben \parencites[151]{celis_2018}{swers_2002}. 
Auf diese Weise wird die Themenwahl verzerrt und eine feministische Agenda abgeschwächt \parencite[151]{celis_2018}. Die bereits in Kapitel \ref{kapitel:geschlechterunterschiede} beschriebene \enquote{male-dominated nature of the political agenda} \parencite[151]{celis_2018} behindert die Inklusion von Frauen und fördert die Stereotypisierungen von Themen, so \textcite{celis_2018}.
Diese Förderung von Stereotypisierungen sowie eine Behinderung der Inklusion von Frauen gilt es zu vermeiden. Hierfür ist eine Analyse von möglichen Themenunterschiede zwischen den Geschlechtern zunächst essentiell. 

\section{Hypothese Unterbrechungen und Rückfragen in den Reden}\label{kapitel:hypothese3}

Eine von \textcite{erikson_2018} durchgeführte Befragung zum Arbeitsumfeld von schwedischen Abgeordneten konnte zeigen, dass weibliche MPs mehr Stress und Druck in ihrem Umfeld ausgesetzt sind und häufiger Opfer negativer Behandlungen im Parlament sind als männliche Abgeordnete. Sie werden eher in den Debatten unterbrochen, sind eher Opfer sexistischer Witze und ihre Äußerlichkeiten wird häufiger negativ kommentiert als die Erscheinung männlicher Kollegen \parencite[13]{erikson_2018}.

Diese Mehrbelastung kann laut \textcite{erikson_2018} zu höheren \enquote{Kosten} für Frauen bezüglich des parlamentarischen Engagements führen - sie müssen mit mehr negativen Erfahrungen rechnen als Männer. Entsprechend könnte dies auch ein Grund für eine geringere Beteiligung von Frauen in Parlamentsdebatten oder generell für die Partizipation von Frauen in politischen Institution darstellen \parencites[vgl.][]{erikson_2018}[vgl.][]{back_2014}.

Es wird erwartet, dass ein solches negatives Verhalten gegenüber Frauen auch im Deutschen Bundestag zu beobachten ist. Entsprechend wird untersucht, wie häufig weibliche und männliche MPs in ihren Reden \emph{negativ} unterbrochen werden. Hierfür werden Reden bezüglich der Anzahl an Zurufen, Gegenrufen, Widerspruch und Lachen über den*die Redner*in untersucht.

\textbf{Hypothese 3a:} \emph{Frauen werden häufiger als Männer während der Rede negativ Unterbrochen}

Eine weitere \emph{negative} Unterbrechung der Render*innen können Rückfragen darstellen - allerdings ist hier eine eindeutige negative Intention nicht anzunehmen. Entsprechend sollen Rückfragen gesondert untersucht werden. Auf Basis der Arbeiten von \textcite{brescoll_2011} und \textcite{eagly_2002} wird erwartet, dass Frauen bei Themen, welche stereotypisch eher nicht mit Frauen verbunden werden, häufiger Inkompetenz unterstellt wird als männliche MPs. Dies kann sich in der Anzahl an Rückfragen äußern, welche während einer Rede gestellt werden. 

\section{Weitere ergänzende Untersuchungen}

Der 19. Deutsche Bundestag war bisher noch nicht Gegenstand einer Untersuchungen der unterschiedlichen Repräsentation von Frauen und Männern bezüglich der Parlamentsdebatten. Im folgenden soll daher ebenso die Anzahl an Reden von weiblichen MPs (nach Fraktion) untersucht werden. \textcite{back_2014} konnte zeigen, dass weibliche MPs im schwedischen Riksdag deutlich weniger Reden halten als ihre männlichen Kollegen. Inwieweit dieses Ergebnis auf den Deutschen Bundestag übertragbar ist, wird im Folgenden geprüft.Aus welchem Grund die Anzahl an Reden relevant ist, verdeutlichen unter anderem \textcite{back_2014}:

\citereset
\begin{quote}
     \enquote{\emph{taking a lot of time} on the floor can in some sense be compared with reaching an important post—in important debates, parties are likely to control the floor agenda} \parencites[507]{back_2014}
\end{quote}

Es kann die Behauptung aufgestellt werden, dass es für die jeweilige Partei von Bedeutung ist, welche Person wie viel Redezeit nutzt \parencite[vgl.][]{proksch_2012} und früheren Studien zufolge wird angenommen, dass persönliche Charaktereigenschaften der MPs einen Einfluss auf die Redezeit haben \parencite[505]{back_2014}. Eine umfassende Analyse des persönlichen Einflusses auf das legislative Verhalten wird an dieser Stelle nicht geleistet (mehr hierzu bei \textcite{saalfeld_2011}). Innerparteiliche Marginalisierungen könnten in Anlehnung an \textcite[507]{back_2014} einen möglichen Grund für weniger weibliche MPs in den Debatten darstellen. Als weiteren Grund wird in der Literatur die Stereotypisierung von Frauen als weniger kompetente Vertreterin aufgeführt oder die in Kapitel \ref{kapitel:geschlechterunterschiede} beschriebene \emph{culture of masculinity} \parencites[507]{back_2014}{lovenduski_2005}. 

Der von \textcite{back_2014} untersuchte schwedische \emph{Riksdag} unterscheidet sich allerdings hinsichtlich der parlamentarischen Gepflogenheiten vom Deutschen Bundestag. Die Redeordnung und Redezeit wird im Bundestag über die Geschäftsordnung und den Ältestenrat festgelegt \parencite[64f.]{linn_2018}. Üblicherweise wird das Modell der \enquote{Berliner Stunde} angewandt -- die Redezeit bemisst sich hierbei über die Stärke der Fraktionen \parencite[vgl.][]{schreiner_2005}. Allerdings wurde im Ältestenrat des 19. Deutschen Bundestages hierzu noch keine Einigung gefunden \parencite[64]{linn_2018}, weshalb die Redezeit im aktuellen Bundestag auf die in der Geschäftsordnung vorgesehenen 15 Minuten begrenzt werden. Bei längeren Reden von Sprecher*innen der Regierung verlängern sich auch die Reden der Oppositionsfraktionen. Generell können die Fraktionen eine längere Redezeit von bis zu 45 Minuten beantragen \parencite{bundestag_2019}. In der \enquote{aktuellen Stunde} des Bundestages sind die Redebeiträge der Abgeordneten auf maximal fünf Minuten begrenzt \parencites[68]{linn_2018}[]{bundestag_2019}.

Bedingt durch die Festlegung der Redezeit durch die Geschäftsordnung und den Ältestenrat sowie eine fehlende Protokollierung der Redezeiten durch den Stenografischen Dienst des Bundestages ist ein Vergleich der Redezeiten von weiblichen und männlichen MPs weder sinnvoll, noch möglich.
\textcite{back_2018} konnte in dem Ländervergleich keine Korrelation zwischen der Anzahl an Frauen in den Fraktionen und der Anzahl an Reden in den Parlamenten feststellen. Dennoch soll in dieser Arbeit ein Überblick über die Anzahl der Reden von Frauen im Vergleich zu den Sitzanteilen in der Fraktion aufgenommen werden -- diese Erkenntnis dient hierbei deskriptivem als deduktivem Erkenntnisgewinn, weshalb für diese Untersuchung auch keine Hypothesen gebildet werden. Die Ergebnisse dieser Untersuchung können die bisherigen Ergebnisse von \textcite{back_2018} ergänzen und überprüfen.\todo[inline]{einem !!!!GGG:?? "hierbei deskriptivem als deduktivem Erkenntnisgewinn, weshalb für diese Untersuchung auch keine Hypothesen gebildet werden"""" Ergibt für mich gerade keinen Sinn, magst du mir das kurz erläutern?!?  - KÖNNEN WIR DAS NICHT ALS KONTROLLVARIABLEN verpacken? GGneu MEINT .... ich verstehe grammatikalisch nicht was deskriptivem als deduktivem er... meint... und Kontrollvariablen meint: ob Unterschieder zwischen m und f ggf nicht allgemein bestehen sondern nur in einigen Parteien... also kontrollieren ob das ergebniss evt. nur auf einige parteien zutrifft....}

\section{Kontrollvariablen}

\begin{itemize}
    \item Geschlecht ist nur eine von vielen Variablen, welche sich beispielsweise auf die Anzahl der Unterbrechungen in einer Bundestagsrede auswirken könnte. Es ist davon auszugehen, dass Fraktionsvorsitzende z. B. während ihrer Reden häufiger unterbrochen werden als Abgeordnete, welche nicht den Vorsitz stellen.
    \item Es handelt sich bei dieser Arbeit um eine x-zentrierte Forschungsfrage \parencite[vgl.][4f.]{ganghof_2005}- inwiefern spielt das Geschlecht eine Rolle?
    \item Mit Kontrollvariablen kann das Risiko einer Scheinkorrelation minimiert werden, da auf mehr Faktoren als nur das geschlecht kontrolliert wird
    \item Neben den Variablen, bei denen ein Einfluss auf die zu überprüfenden abhängigen Variablen erwartet wird (beispielsweise Fraktionsvorsitz, Mitglied der Opposition) sollen auch Variablen, überprüft werden, bei denen keine Auswirkungen auf den Untersuchungsgegenstand erwartet wird -- beispielsweise das Alter des*der Redner*in bezüglich der Anzahl der Unterbrechungen.
    \item Wie die Fragestellung \enquote{Inwieweit unterscheiden sich die Redebeiträge und Verhalten von weiblichen und männlichen Abgeordneten im 19. Deutschen Bundestag bezüglich Häufigkeit, Thematik und Geschlechterneutralität} bereits suggeriert, wird davon ausgegangen, dass die Variable \emph{Geschlecht} einer von vielen Faktoren ist, welche das Verhalten, die Sprache und den Inhalt der Bundestagsdebatte beeinflussen. Entsprechend handelt es sich bei dieser Untersuchung um ein x-zentriertes Forschungsdesign, in welchem auch weitere Faktoren, welche auf die jeweilige abhängige Variable einwirken, untersucht werden sollen  \parencites[vgl.][3f.]{ganghof_2005}.
    \item Überprüft werden mit den Kontrollvariablen die Unterbrechungen und Rückfragen. 
\end{itemize}

\Chapter{Datengrundlage und Methodik}\label{kapitel:datengrundlage}

Seit der 19. Wahlperiode liegen die Protokolle des Deutschen Bundestags als XML-Dateien (Extensible Markup Language) vor und die Struktur der abgelegten Protokolle sind umfassend dokumentiert \parencite{bundestag_2015}. Abgerufen werden können sie über das \emph{Open Data Portal} des Deutschen Bundestags\footnote{\url{https://www.bundestag.de/service/opendata}}. In Verbindung mit den dort abgelegten biografischen Daten aller Bundestagsabgeordneten können Aussagen über die Anzahl und den Inhalt der Reden nach Geschlecht, Alter und Fraktion getroffen werden.

Die Bundestagsprotokolle werden durch den Stenografischen Dienst des Deutschen Bundestages verfasst -- die Stenograf*innen erfassen in den Protokollen neben den Inhalten der Reden auch Zwischenrufe, Beifallsbekundungen, Lachen\footnote{Lachen kann in den Bundestagsprotokollen eher als \enquote{auslachen} oder \enquote{lachen über den Inhalt} verstanden werden - das Lachen beispielsweise über einen Witz wird in Bundestagsprotokollen als \enquote{Heiterkeit} protokolliert.} und weitere Unterbrechungen. Dies ermöglicht eine umfassende Auswertung nach Art der Unterbrechung, da sowohl eine Unterscheidung zwischen positiv intendierten Unterbrechungen (Beifall, Heiterkeit) und (zumeist) negativ intendierten Unterbrechungen (Zwischenrufe, Lachen), als auch die Anzahl an Rückfragen an den*die Redner*in feststellbar ist.

Zur Auswertung und Operationalisierung dienen unterschiedliche Methoden der quantitativen Textanalyse -- die genutzten Methoden sollen in den nachfolgenden Kapiteln kurz erläutert und das Vorgehen dargestellt werden. Die Daten werden für die quantitative Analyse mittels der Programmiersprache R \parencite{rcoreteam_2018} und der \emph{tidyverse}-Paketsammlung \parencite{wickham_2017} aufbereitet und auswertet. Des weiteren werden die Pakete \emph{xml2} \parencite{wickham_2018}, \emph{rvest} \parencite{wickham_2016}, \emph{furrr} \parencite{vaughan_2018}, \emph{stargazer} \parencite{hlavac_2018}, \emph{knitr} \parencite{xie_2014} und \emph{kableExtra} \parencite{zhu_2019} genutzt. 
Alle dieser Arbeit zugrunde liegenden Daten und die zur Auswertung und Aufbereitung entwickelten Skripte sind öffentlich zugänglich und können zur Reproduktion und Überprüfung der Ergebnisse genutzt werden.\footnote{Die Daten können über die Plattform \emph{Github} eingesehen und heruntergeladen werden: \url{https://github.com/holnburger/geschlechterunterschiede_bundestag}}

\section{Überblick der erfassten Reden}\label{kapitel:ueberblick_reden}

Bis zum 12. April 2019 wurden 8.764 Reden im 19. Deutschen Bundestag gehalten -- diese beinhalten auch Reden von Gästen und Regierungsmitgliedern. Allerdings stellt das \emph{Open Data Portal} des Bundestages umfassende (biografische) Daten (beispielsweise Geburtsjahr, Geschlecht, Wahlkreis, Ausschussmitgliedschaften) nur für Abgeordnete bereit. Da nachfolgend die Debatten im Parlament ausgewertet werden und die Reden von Regierungsmitgliedern eher einen berichtenden und die Reden von Gästen einen informellen Charakter besitzen, sollen nachfolgend nur die Redebeiträge von Abgeordneten untersucht werden.

Der Anteil an Reden von weiblichen MPs entspricht mit 31,0\% (vgl. Tabelle \ref{table:uebersicht_reden}) annähernd dem Frauenanteil im Bundestag von 31,8\% (222 Frauen)\footnote{Siehe \url{https://www.bundestag.de/abgeordnete/biografien/mdb_zahlen_19/frauen_maenner-529508}}.
\todo{Fußnote formatieren ohne Lücke, JH: Geht nicht, ist eine URL.}

\begin{table}[htb]
    \centering
    \caption{Anzahl und Anteil der Reden nach Geschlecht der Abgeordneten}
    
\begin{tabular}{lll}
\toprule
Geschlecht & Reden & Anteil\\
\midrule
männlich & 5.327 & 69.2\%\\
weiblich & 2.369 & 30.8\%\\
\bottomrule
\end{tabular}
    \label{table:uebersicht_reden}
\end{table}

Der Anteil der Sitze und die Redeanteile unterscheiden sich bei den Fraktionen teilweise erheblich (vgl. Tabelle \ref{table:vergleich_sitze_reden}). Die AfD hat mit 11,0\% den niedrigsten Frauenanteil im Parlament, der Anteil an Reden von weiblichen MPs beträgt 10,0\% und liegt damit etwas niedriger als der Anteil an Frauen in dieser Fraktion. 
Der größte Unterschied zwischen Anzahl der Sitze von Frauen und Redeanteil findet sich bei der CDU/CSU-Fraktion -- 20,7\% der Fraktion sind Frauen, allerdings halten diese nur 16,0\% der Reden dieser Fraktion. Auch die Linke und die SPD zeigen  einen deutlichen Unterschied zwischen Anteil an Frauen in der Fraktion und den Reden von Frauen -- bei beiden Fraktionen ist der Anteil an Reden niedriger als die Anteil an Frauen. Lediglich bei den Grünen und der FDP findet sich ein höherer Redeanteil von Frauen im Vergleich zu deren Anteil an Sitzen.

Der von \textcite{back_2018} beschriebene \emph{backlash effect}, wonach ein höherer Frauenanteil in einer Fraktion zu weniger Reden von Frauen in den Parlamentsdebatten führt, kann in dieser Erhebung somit nicht bestätigt werden. Die Fraktionen unterscheiden sich diesbezüglich unsystematisch.

\begin{table}[htb]
    \centering
    \caption[Sitz- und Redeanteil von weiblichen MPs nach Fraktionen]{Sitz- und Redeanteil von weiblichen MPs nach Fraktionen. Auswertungszeitraum: 24. Oktober 2017 bis 12. April 2019}
    
\begin{tabular}{lll}
\toprule
Fraktion & \makecell[l]{Sitzanteil\\Frauen} & \makecell[l]{Redeanteil\\Frauen}\\
\midrule
CDU/CSU & 20,7% & 16,0%\\
SPD & 42,8% & 39,0%\\
AfD & 11,0% & 10,0%\\
FDP & 23,7% & 25,8%\\
Die Linke & 53,6% & 50,7%\\
Bündnis 90/ Die Grünen & 58,2% & 61,3%\\
fraktionslos & 25,0% & 43,9%\\
\bottomrule
\end{tabular}
    \label{table:vergleich_sitze_reden}
\end{table}

In der weiteren Auswertung wird nicht auf die Reden von fraktionslosen Abgeordneten eingegangen. Hintergrund hierfür ist sowohl die geringe Anzahl an fraktionslosen Redner*innen im Bundestag (zum Auswertungszeitraum waren lediglich vier der 709 Abgeordneten fraktionslos\footnote{Frauke Petry und Mario Mieruch traten kurz nach der Wahl aus der AfD aus und waren von Beginn des 19. Bundestags nicht Mitglied einer Fraktion. Im November 2018 verlies der Abgeordnete Marco Bülow die SPD-Bundestagsfraktion, im Dezember 2018 trat Uwe Kamann aus der AfD-Fraktion aus.}) als auch Problematiken in der Erhebung -- Marco Bülow und Uwe Kamann hielten ihre Reden sowohl als Mitglieder ihrer Fraktion (SPD bzw. AfD) und später auch als fraktionslose Abgeordnete des Bundestags. Der Redenanteil von Frauen bei fraktionslosen Abgeordneten (vgl. Tablle \ref{table:vergleich_sitze_reden}) ist deshalb stark verzerrt -- zum Teil trifft dies auch auf die Rede- und Sitzanteile bei den Fraktionen der AfD und der SPD zu, allerdings ist diese Verzerrung aufgrund deutlich größeren Anzahl an Reden und Sitzen vernachlässigbar.
Gleichzeitig gelten für fraktionslose Abgeordnete angepasste Regelungen bezüglich Anzahl der Reden und Redezeit im Vergleich zu den übrigen Abgeordneten \parencite[vgl.][583f.]{schreiner_2005}. Wie in Tabelle \ref{table:top_redner} ersichtlich, machen die fraktionslosen Abgeordneten Frauke Petry und Mario Mieruch besonders häufig von ihrem Rederecht Gebrauch.\footnote{Diese Tabelle berücksichtigt nur Bundestagsabgeordnete. Die meisten Reden wurden durch Angela Merkel (126), Heiko Maas (98) und Jens Spahn (76) gehalten.} Dies lässt sich unter anderem durch die insgesamt politisch schwächere Position von fraktionslosen Abgeordneten erklären \parencite[372]{morlok_2018} -- da sie kein Antragsrecht und keine Stimmrecht und Rederecht in Ausschüssen besitzen \parencite[vgl.][]{morlok_2018}, nutzen diese vor allem ihr Rederecht im Plenum.

Da sich das Redeverhalten sowie die Rechte von fraktionslosen Abgeordneten zu den Abgeordneten in Fraktionen derart unterscheidet, werden diese in der weiteren Auswertung nicht berücksichtigt, um eine Verzerrung der Ergebnisse zu vermeiden. Es werden insgesamt 8.186 Reden von insgesamt 677 Abgeordneten\footnote{Im Untersuchungszeitraum haben noch nicht alle Abgeordneten mindestens ein Mal eine Rede gehalten -- entsprechend ergibt sich die Differenz zwischen den insgesamt 709 Abgeordneten und der Anzahl an Abgeordneten in diesem Datensatz.} untersucht. 

\begin{table}[htb]
    \centering
    \caption[Abgeordnete mit den meisten Reden im 19. Deutschen Bundestag.]{Abgeordnete mit den meisten Reden im 19. Deutschen Bundestag. Auswertungszeitraum: 24. Oktober 2017 bis 12. April 2019}
    
\begin{tabular}{lrr}
\toprule
Name & Fraktion & Reden\\
\midrule
Volker Ullrich & CDU/CSU & 68\\
Mario Mieruch & fraktionslos & 55\\
Frauke Petry & fraktionslos & 50\\
Alexander Graf Lambsdorff & FDP & 48\\
Katja Keul & BÜNDNIS 90/DIE GRÜNEN & 38\\
Johann Saathoff & SPD & 37\\
Sebastian Brehm & CDU/CSU & 37\\
Jürgen Hardt & CDU/CSU & 36\\
Helge Lindh & SPD & 36\\
Fabio De Masi & DIE LINKE & 36\\
\bottomrule
\end{tabular}
    \label{table:top_redner}
\end{table}

\section{Quantitative Textanalyse}

Die stetig zunehmende Rechenleistung, die ständige Weiterentwicklung von Methoden der quantitativen Textanalyse sowie die immer größer werdenden, zu analysierenden Datenmengen\footnote{Ein besonders prägnantes Beispiel ist hier der \emph{Manifesto Corpus}. Mit diesem Projekt wurden 1.800 maschinenlesbare Dokumente politischer Parteien aus 40 verschiedenen Ländern zur quantitativen Textanalyse veröffentlicht \parencite{merz_2016}.} haben in den vergangenen Jahren zu einem stetig steigenden Interesse der Politikwissenschaft an der Auswertung von Text als Daten (\enquote{\emph{text-as-data}}) geführt \parencite[vgl.][]{wilkerson_2017}.\footnote{Eine Zusammenfassung der zahlreichen Methoden der automatisierten, quantitativen Texterfassung und eine Beleuchtung der Möglichkeiten und auch Fallstricke dieser Methoden findet sich bei \textcite{grimmer_2013}}. 

Auch in dieser Arbeit werden diese weiterentwickelten Methoden genutzt, um die in Kapitel \ref{kapitel:ueberblick_reden} bereits angedeuteten umfassenden Datenmengen \todo{Datenmengen nochmal nennen. JH: Wie meinst du das?} erfassbar und auswertbar zu machen.

Hierbei soll unter anderem auf die Möglichkeiten von diktionärbasierten Textauswertungen und die automatisierte Erfassung von Themen (das sogenannte \emph{automated topic modeling}) zurückgegriffen werden -- beide Methoden der quantitativen Textanalyse werden in den nachfolgenden Kapiteln vorgestellt und anhand des Untersuchungsgegenstandes sowie dessen Operationalisierung genauer erläutert werden.


\subsection{Diktionärbasierte Textauswertung}\label{kapitel:diktionär}

Mittels sogenannter Diktionäre\footnote{In der überwiegend englischsprachigen Literatur wird von \emph{ditionaries} gesprochen. In Ermangelung eines besseren Begriffes, wird in der vorliegenden Arbeit der Begriff \enquote{Diktionär} gewählt, alternativ könnte hier auch von Wörterbüchern gesprochen werden.} können Texte einfach klassifiziert werden \parencite[274]{grimmer_2013}. In Diktionären können Wörter oder Phrasen beispielsweise mit Themen oder Stimmungen verknüpft werden -- über einfaches Zählen von Wörtern in einem Dokument können anschließend Aussagen über dieses vorgenommen werden.
So kann etwa überprüft werden, ob in Zeitungsberichten oder Parteiveröffentlichungen ein Thema eher negativ oder positiv besetzt ist und daraus Rückschlüsse über die öffentliche Berichterstattung und Rezeption der Parteien gezogen werden \parencites[vgl.][]{haselmayer_2017}[vgl.][]{backfried_2016}. Allerdings ist hierbei zu beachten, dass eine diktionärbasierte Textanalyse sehr stark kontextabhängig ist -- ein Diktionär sollte immer einen Bezug auf die zu untersuchende Domäne aufweisen und außerhalb dieser Domäne nicht genutzt werden \parencite[268]{grimmer_2013}. Ein Diktionär, welches zur Bestimmung der Tonalität von Reden im Parlamenten ausgearbeitet wurde (beispielsweise das Diktionär von \textcite{haselmayer_2017}) kann nicht einfach auf auf die Analyse von Beiträgen auf den Sozialen Netzwerken herangezogen werden -- die Syntax, Semenatik und das Vokabular unterscheidet sich hier deutlich zwischen den Domänen \parencite[vgl.][534f.]{wilkerson_2017}.

Im Rahmen dieser Arbeit soll ein diktionärbasierter Ansatz zur Untersuchung der Nutzung von genderinklusiver Sprache (Hypothese 1 -- siehe Kapitel \ref{kapitel:hypothese1}) herangezogen werden. Ein solcher Einsatz eignet sich ideal, um diese Hypothese zu überprüfen. Mittels eines Wörterbuchs bestehend aus genderinklusiven und genderexklusiven Wörtern und Phrasen kann hierbei untersucht werden, ob weibliche MPs in Bundestagsreden häufiger auf genderinklusive Wörter und Formulierungen zurückgreifen als männliche MPs.  

Allerdings konnte in der umfassenden Recherche im Vorfeld dieser Arbeit kein wissenschaftlich erarbeitetes Wörterbuch gefunden werden, welches genderinklusive und genderexklusive Begrifflichkeiten und Formulierungen enthält. Idealerweise wird ein solches Wörterbuch entweder interdisziplinär erarbeitet. Das \emph{SentiWS}-Wörterbuch von \textcite{remus_sentiws_2010} wurde beispielsweise mit Beteiligung von Informatiker*innen und Linguist*innen entwickelt. 
Ein anderer Ansatz ist die Erarbeitung durch eine Vielzahl an Beteiligten, welche unabhängig voneinander Bewertungen beispielsweise zu negativer oder positiver Konotation von Wörtern und Sätzen abgeben -- ein solches Vorgehen wurde über \emph{crowdcoding} beispielsweise bei \textcite{haselmayer_2017} realisiert.

Beide Ansätze können im Rahmen dieser Arbeit nicht realisiert werden, da die entsprechenden Ressourcen nicht zur Verfügung stehen. Um dennoch ein geeignetes Diktionär zu entwickeln, wurde an dieser Stelle auf andere Möglichkeiten zurückgegriffen:
Ein ehrenamtlich gepflegtes \emph{Genderwörterbuch} auf der von Johanna Usinger und Philipp Müller betriebenen Webseite \url{www.geschicktgendern.de} pflegt durch Unterstützung von Freiwilligen \enquote{gendergerechte Alternativen} zu \enquote{üblchen Begriffen} ein und erweitert sich ständig. Im Rahmen dieser Arbeit wurde dieses Online-Wörterbuch mittels sogenanntem \emph{web scraping} \parencite{wickham_2016} abgegriffen und anschließend manuell als Diktionär aufbereitet. 

Hierbei mussten zahlreiche Einträge entfernt werden, da sie für die zu untersuchende Domäne zu spezifisch (etwa die genderexklusive Formulierung \enquote{… Um an unserer Klinik als Arzt arbeiten zu können, müssen Sie die Approbation im Original vorlegen.}) oder zu unspezifisch (\enquote{Eltern} als genderinklusive Variante von \enquote{Mütter}) erschienen. Einige genderexklusive Wörter wurden noch ohne genderinklusive Variante aufgeführt (\enquote{noch kein passender Begriff gefunden; senden Sie Ihren Vorschlag über das Kontaktformular}) -- auch diese wurden aus der Erhebung entfernt. Ein Auszug des Diktionärs über genderinklusive und -exklusive Wörter findet sich in Tabelle \ref{table:beispiele_woerter}. Der Tabellenauszug zeigt ebenfalls, dass einige genderinklusive Alternativen zu den jeweiligen Begriffen durchaus problematisch sein können -- die Gleichsetzung von \enquote{Schüler} und \enquote{Kinder} ist nicht immer möglich und manchmal auch falsch. Entsprechend muss dies bei der weiteren Auswertung und der Kritik berücksichtigt werden.


\begin{table}[htb]
    \centering
    \caption[Auszug an genderexklusiven und -inklusiven Begriffen nach Aufbereitung der Daten]{Auszug an genderexklusiven und -inklusiven Begriffen nach Aufbereitung der Daten}
    
\begin{tabular}{ll}
\toprule
Übliche Begriffe & Genderinklusive Varianten\\
\midrule
Unternehmer & selbstständige Person\\
Verbraucher & Konsumierende\\
Fahrzeughalter & eine ein Fahrzeug besitzende Person\\
Sozialpädagoge & sozialpädagogische Fachkraft\\
Mutterschutz & Elternschutz\\
Lehrling & lernende Person\\
Protagonist & zentrale Gestalt\\
Befragter & Person, die befragt wurde\\
Organisatoren & Organisationsteam\\
Polizisten & Polizei\\
\bottomrule
\end{tabular}
    \label{table:beispiele_woerter}
\end{table}

Die in Kapitel \ref{kapitel:gfl-studien} und \ref{kapitel:hypothese1} behandelte \emph{gender-fair language} umfasst jedoch mehr als nur einzelne Begriffe. In dieser Arbeit soll auch herausgefunden werden, ob Abgeordnete in ihren Reden eine genderinklusive Ansprache wählen -- beispielsweise, ob von \enquote{Kolleginnen und Kollegen} anstatt exklusiv von \enquote{Kollegen} oder \enquote{Kolleginnen} gesprochen wird\footnote{Der Stenografische Dienst des Bundestages protokolliert leider nicht, ob Abgeordnete mit einem sogenannten \emph{Gender-Gap} sprechen -- also das bewusste Pausieren vor der geschlechterspezifischen Endung, etwa bei \emph{Kolleg\_innen} \parencite[vgl.][]{reisigl_2017}. Gender-Gaps werden in den Bundestagsprotokollen als rein weibliche Aussprache protokolliert.}. 

Mittels sogenannter \emph{regular expressions} \parencite{thompson_1968} können die Protokolle nach solchen genderinklusiven Formulierungen untersucht werden -- hierfür wird mit Hilfe von spezifisch definierten Regeln nach diesen Passagen im Datensatz gesucht. In dieser Arbeit wurde nach Passagen gesucht, welche alle folgende Regeln erfüllen:
\begin{itemize}
    \setlength\itemsep{1em}
    \item Zwei Wörter, welche durch ein \enquote{und} oder \enquote{oder} getrennt werden.
    \item Eines dieser Wörter endet mit den Buchstaben \enquote{in} oder \enquote{innen}, das zweite Wort endet hingegen mit \enquote{e}, \enquote{er} oder \enquote{en}.
    \item Die Wörter bestehen aus jeweils mindestens vier Buchstaben vor den jeweiligen Endungen.
\end{itemize}

Mit diesen Regeln konnten zunächst 507 Formulierungen identifiziert werden, welche nach einer manueller Bereinigung auf 517 genderinklusive Formulierungen gekürzt wurden. Die zehn häufigsten Ansprachen, welche diesen Regeln entsprechen, sind in Tabelle \ref{table:top_gender_phrasen} aufgeführt. 

\begin{table}[htb]
    \centering
    \caption[Häufigste genderinklusive Ansprachen in den Reden des 19. Deutschen Bundestages]{Häufigste genderinklusive Ansprachen in den Reden des 19. Deutschen Bundestages. Die Kleinschreibung ist technisch bedingt.}
    
\begin{tabular}{lr}
\toprule
Forumulierung & n\\
\midrule
kolleginnen und kollegen & 7677\\
bürgerinnen und bürger & 858\\
soldatinnen und soldaten & 639\\
arbeitnehmerinnen und arbeitnehmer & 195\\
verbraucherinnen und verbraucher & 173\\
mitarbeiterinnen und mitarbeiter & 163\\
mieterinnen und mieter & 148\\
ärztinnen und ärzte & 117\\
schülerinnen und schüler & 99\\
patientinnen und patienten & 97\\
\bottomrule
\end{tabular}
    \label{table:top_gender_phrasen}
\end{table}

Auf diese Weise wurde für diese Arbeit ein Diktionär entwickelt, welches genderexklusive Begriffe und genderinklusive Varianten sowie Formulierungen enthält. Kritisch ist hierbei die bereits beschriebene Datengrundlage zu bewerten -- da diese nicht wissenschaftlich entwickelt sondern durch einem ehrenamtliches Projekt entstammen, sind die Ergebnisse und Rückschlüsse der Arbeit, welche auf diesem Diktionär basieren, entsprechend mit Vorsicht zu betrachten.
Für zukünftige weitere Erhebungen benötigt es ein möglichst interdisziplinäre entwickeltes Diktionär, welches für die Untersuchung deutschsprachiger (politischer)\todo[inline]{politischer - streichen, oder?}\todo[inline]{bezieht sich auf die Domäne "Politik"} Texte und Reden geeignet ist. Eine Grundlage oder Inspiration hierfür könnte das im Rahmen dieser Arbeit entwickelte Diktionär mit insgesamt 427 genderinklusiven Formulierungen, 1.059 genderinklusiven und 605 genderexklusiven Begriffen sein (vgl. Tabelle \ref{table:zusammenfassung_dict}.

\begin{table}[htb]
    \centering
    \caption[Zusammenfassung der Daten des entwickelten Diktionärs]{Zusammenfassung der Daten des entwickelten Diktionärs}
    
\begin{tabular}{rr}
\toprule
Übliche Begriffe & Genderinklusive Begriffe\\
\midrule
699 & 1242\\
\bottomrule
\end{tabular}
    \label{table:zusammenfassung_dict}
\end{table}

Mittels des Diktionärs kann nun der Anteil genderinklusiver Begriffe und Formulierungen in den jeweiligen Reden untersucht und Hypothese 1 (siehe Kapitel \ref{kapitel:hypothese1}) überprüft werden. Es ist allerdings zu berücksichtigen, dass ein bloßes zählen geschlechterinklusiver Begriffe und Formulierungen nicht ausreichend für die Hypothesenprüfung ist:
Die unterschiedliche Länge der Reden bedingt, dass längere Reden vermutlich mehr geschlechterinklusive Passagen enthalten. Sehr kurze Reden\footnote{Beispielsweise sei hier die Wahl des Bundestagspräsidenten bei der ersten Sitzung des 19. Bundestages genannt: \enquote{Der Kollege Dr. Wolfgang Schäuble ist vorgeschlagen. – Ich darf Sie fragen: Sind Sie bereit, zu kandidieren?} Dr. Wolfgang Schäuble (CDU/CSU): \enquote{Ja, Herr Präsident.}. Die Antwort Wolfgang Schäubles wird als Rede in den digitalen Protokollen des Bundestages vermerkt.} enthalten hingegegen vermutlich keine genderinklusiven Begriffe und Formulierungen. Aufgrund dieser Umstände wird der Anteil an geschlechterinklusiven Wörtern im Verhältnis zur Länge der Reden untersucht. 
Reden, welche kürzer als 100 Wörter sind (460 von den untersuchten 81.86 Reden) werden in der Analyse nicht berücksichtigt, da diese sich im Charakter von der übrigen Plenardebatte deutlich unterscheiden. Hierbei handelt es sich zumeist um Antworten an das Präsidium oder Rückfragen an Redner*innen.

Mittels einer Boxplot-Grafik können die Mittelwerte und die Streuung der Anteile genderinklusiver Begriffe sowie die Formulierungen in den Reden anschließend visualisiert werden. Ein t-Test \parencite[vgl.][164ff.]{diaz-bone_2018} ermöglicht anschließend eine Hypothesenprüfung gegen die Nullhypothese $H_0$ (\emph{Frauen und Männer verwenden in ihren Reden gender-fair language gleich häufig.}).

\subsection{Topic modeling}

\subsection{Erfassung von Unterbrechungen und Rückfragen}


\Chapter{Ergebnisse und Auswertungen}

\section{Genderinklusive Sprache}


\begin{itemize}
    \item Wie umfänglich dargelegt, wird vermutet, dass weibliche MPs häufiger als ihre männlichen Kollegen Gebrauch von genderinklusiven Begriffen und Formulierungen (siehe Kapitel \ref{kapitel:gfl-studien} und \ref{kapitel:diktionär} machen und dies auch entsprechend in den Parlamentsdebatten des Deutschen Bundestages zu beobachten ist.
    \item Eine Überprüfung mittels des für diese Arbeit eigens entwickelten Diktionärs (vgl. Kapitel \ref{kapitel:diktionär}) kann gezeigt werden, dass weibliche MPs häufiger genderinklusive Begriffe und Formulierungen nutzen (siehe Abb. \ref{fig:boxplot_gfl})
\end{itemize}

\begin{figure}
    \centering
    \includegraphics[width=0.7\textwidth]{document/images/boxplot_gfl.pdf}
    % Der Teil in eckigen Klammern ist der Kurztitel für das Table of Contents
    \caption[Anteil geschlechterinklusiver Begriffe und Formulierung pro Rede im Deutschen Bundestag]{Anteil geschlechterinklusiver Begriffe und Formulierung pro Rede im Deutschen Bundestag. Quelle: Eigene Erhebung}
    \label{fig:boxplot_gfl}
\end{figure}

\begin{itemize}
    \item Das Ergebnis des t-Test (\textit{t}(4263.51)~=~3.63, \textit{p}~<~.001, \textit{d}~=~0.09) zeigt, dass die Nullhypothse $H_0$ (\emph{Frauen und Männer verwenden in ihren Reden gender-fair language gleich häufig.} zugunsten der Alternativhypothese $H_1$ (\emph{Frauen verwenden in ihren Reden häufiger gender-fair language als Männer} abgelehnt werden kann.
\end{itemize}

\section{Themen der Reden}

\section{Unterbrechungen und Rückfragen}

\Chapter{Methodenkriktik und Forschungsausblick}

\Chapter{Diskussion}\label{kapitel:diskussion}

\begin{quote}
    \enquote{Although women remain significantly under-represented in today’s parliaments, they are now looking beyond the numbers to focus on what they can actually do while in parliament — how they can make an impact, whatever their numbers may be. They are learning the rules of the game and using this knowledge and understanding to promote women’s issues and concerns from inside the world’s legislatures. In so doing, they are not only increasing the chances of their own success, but they are also paving the way for a new generation of women to enter the legislative process.} \parencite[3]{lovenduski_2015}
    
    
    \citereset
\begin{quote}
    \enquote{Gender equal representation is not only about the proportion of men and women legislators or the outcomes of politics}\parencite[197]{erikson_2018}
\end{quote}



\end{quote}
\textbf{Hypothese 3b:} \emph{Frauen erfahren häufiger Rückfragen in ihren Reden, wenn wenige andere Frauen zu diesem Thema gesprochen haben}\footnote{Diese Hypothese kann nur in Verbindung mit der Hypothese 3 zur Thematisierung in den Reden überprüft werden.}
\todo[inline]{Ich bin mir noch nicht sicher, ob wir die tatsächlich untersuchen werden können. Ich stoße da ziemlich auf Probleme. Ich würde sonst Vorschlagen: Streichen und in die Kritik mit rein. Wir werden nicht sehen, dass Frauen eher unterbrochen werden -- vielleicht werden ihnen aber eher Rückfragen und damit Inkompetenz unterstellt. Unsere Arbeit hat dafür einen ersten Teil der Analyse gestellt und weitergehende Forschung dazu ist notwendig. !!!GG: Stimme zu}

\setstretch{1.0}
\printbibliography[title={Literaturverzeichnis}]

\end{document}

\chapter{Beispiele LaTeX}

Hier eine Tabelle, welche im Ordner \enquote{table} abgespeichert wurde und mit folgendem Befehl eingelesen werden kann. Verweise siehe Tabelle \ref{table:unterbrechung_stats}. Außerdem möchte ich hier noch auf die Abbildung \ref{fig:lda_example} auf Seite \pageref{fig:lda_example} verweisen. 

\begin{table}[htb]
    \centering
    \caption{Statistiken zu Unterbrechungen in den Bundestagsreden}
    
\begin{tabular}{lrrrrrr}
\toprule
Geschlecht & min & max & mean & sd & n & var\\
\midrule
männlich & 0 & 41 & 3.964055 & 4.857574 & 4173 & 23.59602\\
weiblich & 0 & 39 & 3.276607 & 4.110263 & 1851 & 16.89426\\
\bottomrule
\end{tabular}
    \label{table:unterbrechung_stats}
\end{table}

\section{Unterkapitel}

\begin{figure}
    \centering
    \includegraphics[width=0.8\textwidth]{document/images/lda_topic_model.png}
    % Der Teil in eckigen Klammern ist der Kurztitel für das Table of Contents
    \caption[Schematische Darstellung eines LDA Topic Modelings]{Schematische Darstellung eines LDA Topic Modelings. Quelle: \citetitle{blei_2012} \parencite{blei_2012}}
    \label{fig:lda_example}
\end{figure}

% Wenn wir nicht wollen, dass ein Kapitel auf einer neuen Seite startet, nutzen wir folgenden Befehl hier:
{\let\clearpage\relax \chapter{Theoriefindung}}



Die Überprüfung der \textbf{Hypothesen 1, 2a und 2b} sollen mittels Regressionsanalysen untersucht werden. Neben dem Geschlecht des*der Redner*in sollen folgende Faktoren als Kontrollvariablen erhoben und untersucht werden: (KRITIK HIERBEI; MUSS NACH OBEN; NICHT ERST HIER! GEHÖRT ZUR KONZEPTION)

\textbf{der*die Rednerin ist Mitglied der Oppositions- bzw. Regierungsfraktion}

Es wird erwartet, dass Mitglieder der Regierungsfraktion deutlich häufiger mit Rückfragen konfrontiert werden, da sie eine stärkere Einbindung in den Legislativprozess erfahren und die Kontrolle der Regierung zu den Kernaufgaben der Opposition zählen.

\textbf{der*die Redner*in ist Mitglied des Fraktionsvorsitzes}

Es wird erwartet, dass Mitglieder des Vorsitzes der Fraktionen beispielsweise häufiger unterbrochen werden. Zum einen, da Reden, welche auf Antrag der Opposition länger gehalten werden, häufig von den Fraktionsvorsitzenden gehalten werden, aber auch, da diese Reden häufig bei volleren Plenarsälen gehalten werden.

\textbf{der*die Redern*in ist Mitglied der Fraktion "Alternative für Deutschland"}

Hier wird erwartet, dass Abgeordnete der AfD-Fraktion deutlich seltener GFL nutzen, wohingehend die Anzahl der Unterbrechungen deutlich höher erwartet wird. Populistische Parteien setzen sich unter anderem durch provokante Thesen und ein provokantes Auftreten von anderen Parteien ab \parencites[vgl.][]{decker_populismus_2006}{priester_2012}. Dies lässt vermuten, dass mehr Zwischenrufe von anderen Abgeordneten in den Reden der AfD-Mitglieder geäußert werden.

\textbf{die Anzahl der bisherigen Bundestagsmandate des*der Redner*in}

Bei dieser Variable wird kein Einfluss auf die abhängigen Variablen (Nutzung von GFL, Anzahl der Unterbrechungen, Anzahl der Rückfragen) erwartet, sie dient der Überprüfung der Validität der Methoden.

\textbf{das Alter des*der Redner*in}

Hier wird keine Einfluss auf die abhängigen Variablen erwartet. Diese Variable dient der Überprüfung der Validität der Methode.

Die {Hypothesen 1, 2a und 2b} werden über eine Regressionsanalyse überprüft. Die Variable Geschlecht wird hierbei als Dummy-Variable kodiert. Der Einfluss der Kontrollvariablen etwa auf die abhängige Variable \emph{Anzahl der Unterbrechungen} wird ebenfalls überprüft. Da es sich bei den abhängigen Variablen um Zähldaten handelt, wird vermutlich eine negative Binomialregression angewandt (welche auch bei \textcite{back_2014} angewandt wurde) oder eine OLS-Regression untersucht (angewandt bei \textcite{coffe_2013}).


\textbf{Hypothese 3} zur Untersuchung der Themen in den Reden der Abgeordneten soll über ein automatisiertes \emph{Topic Modeling} untersucht werden. Anders als bei \textcite{back_2014}, welche die Themen der Reden der Abgeordneten anhand der vorangegangen Reden von Minister*innen untersucht hat, soll durch ein automatisiertes \emph{Topic Modeling} einerseits einen geschlechterstereotypen Bias verhindern aber auch eine genauere Zuordnung an Themen ermöglichen. 
Automatisierte \emph{Topic Modelings} wurden bereits auf Reden des Europaparlaments angewandt, um Agenda-Trends zu identifizieren \parencite[vgl.][]{greene_2016} und werden auch in der vergangenen Studie von Bäck und Debus zur Identifikation von geschlechtsbezogener \emph{Topics} vorgeschlagen \parencite*[18]{back_2018}.

Als automatisiertes \emph{Topic Modeling Framework} wird das von der University of Cambridge ausgezeichnete\footnote{https://www.cambridge.org/core/membership/spm/about-us/awards/statistical-software-award/statistical-software-award-18} \emph{Structural Topic Modeling} von \textcite{roberts_2018} angewendet. Dies ermöglicht es, auch innerhalb von Reden mehrere Inhalte zu identifizieren und erlaubt so ein genaueres Vorgehen bei der Untersuchung möglicher geschlechterbezogener Themen. In unserem Datensatz sind \textbf{XXX} Reden enthalten.



%% Bibliography

\setstretch{1.0}
\printbibliography[title={Literaturverzeichnis}]

\end{document}