% ---------------------------------------------------------------------
% Das Dokument kompiliert mit pdflatex und ist auf Basis 
% von Koma-Script entstanden. 
%
% Autor des Templates (für Anmerkungen): 
% Michael von Riegen, riegen@informatik.uni-hamburg.de
%
% Einzelne Code-Teile für das Titelblatt sind aus dem Template 
% von Benjamin Kirchheim entnommen.
% Neue Titelseite von Josef Holnburger
%
% 25.05.09, Frank Langanke: Vorlage auf aktuelle KOMA-Version aktualisiert
% 26.05.09, Michael von Riegen: Anmerkung --> aktuelles Koma-Script ist nötig!
% 17.10.2016 neues Uni logo
% 21.08.2018 biblatex und scrlayer-scrpage statt scrpage2, anpassen auf SoWi-Layout
% ---------------------------------------------------------------------

% Wir nutzen Biblatex und biber für das Literaturverzeichnis
\documentclass[12pt, twoside = false, bibliography=totoc]{scrbook}
\usepackage[backend=biber, style=authoryear]{biblatex}
\addbibresource{Bib.bib}

% Daten der Hausarbeit
% Bitte füllen Sie die folgenden Daten vollständig aus
\newcommand\myName{Josef Holnburger}
\newcommand\myStreetAddress{XXX}
\newcommand\myCityAddress{XXX}
\newcommand\myEmail{josef@holnburger.com}
\newcommand\myKeywords{bundestag, gender, geschlecht, partizipation, topic model, gfl, gender fair language} % Optional
\newcommand\myMatNr{XXX}
\newcommand\myTitle{Geschlechterunterschiede in den Reden des Deutschen Bundestages}
\newcommand\mySubTitle{Auswertung des 19. Deutschen Bundestages}
\newcommand\thesisType{Projektarbeit}
\newcommand\fachbereich{Sozialwissenschaften} 
\newcommand\fachgebiet{Politikwissenschaft}
\newcommand\courseOfStudies{Forschungskolloquium Vergleichende und Regionalstudien}
\newcommand\supervisorType{Dozenten} % Dozent*in, Seminarleiter etc.
\newcommand\supervisor1{Prof. Kai-Uwe Schnapp}
\newcommand\supervisor2{PD Dr. Falk Daviter}
\newcommand\currentSemester{Wintersemester 2018/19}
\newcommand\dateOfSubmission{XX.XX.XXXX}

% Import von Paketen und Optionen die das gesamte Dokument betreffen
% sind in myPreamble.sty ausgelagert.
\usepackage{myPreamble}
   
\begin{document}


% TITELSEITE
% *************************************************************************
% *    Thesis / Dissertation Latex Template                                        
% *    
% *    Author: Leonard Heilig <leonard.heilig@uni-hamburg.de>
% *    modified by: Josef Holnburger <josef@holnburger.com>
% *   
% *    Note: some parts of this template are based on the VSIS template
% *              of Michael von Riegen <riegen@informatik.uni-hamburg.de>
% *   
% *************************************************************************

\begin{titlepage}

% START PAGE: -1
\setcounter{page}{-1}    

\begin{minipage}[t]{0.6\textwidth}
\setstretch{1.0}
\flushleft 
			Universität Hamburg \\
			Fachbereich: Sozialwissenschaften \\
			Fachgebiet: Politikwissenschaft \\
			Seminar: Forschungsseminar Vergleichende und Regionalstudien \\ 
			Dozenten: Prof. Kai-Uwe Schnapp \\
			PD Dr. Falk Daviter \\
			Wintersemester 2018/19 \\
\end{minipage}
\hfill
\begin{minipage}[t][1.7cm][b]{0.35\textwidth}
\noindent\includegraphics[width=\textwidth]{images/UHH-Logo_2010_Farbe_CMYK.pdf}
\end{minipage}

\vspace*{\fill}
\begin{center}
	% THESIS TYPE
	\vspace{1cm}\noindent {\textbf{\thesisType}} \vspace{0.2cm} \\
	% THESIS TITLE
	\textbf{\Large \myTitle} \\
	\textbf{\mySubTitle} \\
	\dateOfSubmission \\
\end{center}
\vspace*{\fill}

\begin{minipage}[t]{0.48\textwidth}
\setstretch{1.0}
    \flushleft 
    Gina-Gabriela Görner \\
    Matrikelnummer: XXX \\
    XXX \vspace{0.1cm} \\ 
	XXX \vspace{0.1cm}  \\
	E-Mail: XXX \\ 
\end{minipage}
\begin{minipage}[t]{0.48\textwidth}
	\flushleft
	\setstretch{1.0}
	Josef Holnburger \\
	Matrikelnummer: XXX \\
	XXX \vspace{0.1cm} \\
	XXX \vspace{0.1cm} \\
	E-Mail: josef@holnburger.com \\
\end{minipage}


\end{titlepage}



% VERZEICHNISSE (Inhaltsverzeichnis, Abkürzungen)
% Vorspann einleiten --> Seitennummerierung römisch
\frontmatter

% Inhaltsverzeichnis
\tableofcontents

% Abbildungs- und Tabllenverzeichnis auf einer Seite
\listoffigures
\addcontentsline{toc}{chapter}{\listfigurename}
\vspace*{24pt}
{\let\clearpage\relax \listoftables}	
\addcontentsline{toc}{chapter}{\listtablename}

% Hauptteil einleiten --> Seitennummerierung wieder arabisch
\mainmatter

\chapter{Einleitung}\label{Einleitung}

Hier kommt eine Quellenangabe \parencite{bartlett_populism_2014}. Weiterhin sieht man hier einen Link auf Abbildung \ref{fig:Übersicht_Parteien_2013-17}. Lorem ipsum dolor sit amet, consectetuer\footnote{Hier ist eine Fußnote!} adipiscing elit \parencite[vgl.][]{bobba_age_2017}. 
\blindtext

\vspace*{4pt}
\begin{figure}[h]
	\centering
		\includegraphics[width=1.0\textwidth]{plots/overview_posts_user_2017}
	\caption[Übersicht der Likes, Anzahl der Beiträge, Kommentare und kommentierenden Nutzer auf den Facebookseiten der Parteien]{Übersicht der Likes, Beiträge, Kommentare und kommentierenden Nutzer der Facebookseiten der Parteien. Auswertungszeitraum: 01.01.2017 bis 31.12.2017}
	\label{fig:Übersicht_Parteien_2013-17}
\end{figure}

\blindtext

\blindtext

\blindtext \parencite[vgl.][S.12]{forchtner_mediatization_2013}

\blinddocument

\section{Facebook und die Parteien}

\blindtext

\captionsetup{justification=raggedright}
\input{tables/Reactions} 

\chapter{Reaktionen}

\blindtext


\printbibliography[title={Literaturverzeichnis}]

    
\end{document}