\documentclass[12pt, 
    twoside=false, 
    bibliography=totoc, 
    numbers=endperiod, 
    headings=normal, 
    toc=chapterentrydotfill
    ]{scrbook}
\usepackage[utf8]{inputenc}
\usepackage[ngerman]{babel}
\usepackage{blindtext}
\usepackage{csquotes}
\usepackage{booktabs}
\usepackage{setspace}
\usepackage{mathpazo}
\usepackage{graphicx}
\usepackage{etoolbox}
\usepackage[
  	pdfstartview=FitH,   
  	pdffitwindow=true,
  	colorlinks,
  	linkcolor=black,
  	anchorcolor=black,
  	citecolor=black,
  	urlcolor=black
  	]{hyperref}
\usepackage[labelfont=bf]{caption}
\usepackage{float}
\usepackage[
    backend=biber, 
    style=authoryear-ibid, 
    eprint=false,
    url=false,
    doi=false,
    isbn=false,
    dashed=false
    ]{biblatex}
\addbibresource{Bib.bib}

% Layoutanpassungen
\setkomafont{sectioning}{\normalcolor\bfseries}
\renewcommand*{\chapterheadstartvskip}{\vspace*{-\topskip}}
\KOMAoptions{headsepline = true}
\setlength{\textheight}{1.05\textheight}
\AtBeginEnvironment{quote}{\singlespacing\small}

\begin{document}

\begin{titlepage}
    \begin{minipage}[t]{0.6\textwidth}
    \flushleft 
    Universität Hamburg \\
    Fachbereich: Sozialwissenschaften \\
    Fachgebiet: Politikwissenschaft \\
    Seminar: Forschungsseminar Vergleichende und Regionalstudien \\ 
    Dozenten: Prof. Kai-Uwe Schnapp \\
    PD Dr. Falk Daviter \\
    Wintersemester 2018/19 \\
    \end{minipage}
    \hfill
    \begin{minipage}[t][1.7cm][b]{0.35\textwidth}
    \includegraphics[width=\textwidth]{images/UHH-Logo_2010_Farbe_CMYK.pdf}
    \end{minipage}
    
    \vspace*{\fill}
    \begin{center}
	\vspace{1cm}\noindent {\textbf{Projektarbeit}} \vspace{0.2cm} \\
	\textbf{\Large Geschlechterunterschiede} \\
	\textbf{WIP} \\
	\vspace{0.2cm}
	30.03.2019
	\end{center}
    \vspace*{\fill}
	
	\begin{minipage}[t]{0.48\textwidth}
    \flushleft 
    Gina-Gabriela Görner \\
    Matrikelnummer: XXX \\
    XXX \vspace{0.1cm} \\ 
	XXX \vspace{0.1cm}  \\
	E-Mail: XXX \\ 
    \end{minipage}
    \begin{minipage}[t]{0.48\textwidth}
	\flushleft
	Josef Holnburger \\
	Matrikelnummer: XXX \\
	XXX \vspace{0.1cm} \\
	XXX \vspace{0.1cm} \\
	E-Mail: josef@holnburger.com \\
    \end{minipage}

\end{titlepage}

\frontmatter

\tableofcontents

\mainmatter

\setstretch{1.5}

\chapter{Einleitung}

\begin{itemize}
    \item Keine \emph{detective story}
    \item Kurze Begründung der Relevanz des Forschungsgebiets
    \item Kurze Vorstellung der Forschungsfrage und der Besonderheit des Datensatzes und der Methodik
    \item Was hebt diese Arbeit von anderen Arbeiten ab
    \item Kurzes anreissen erster Ergebnisse (finde ich besser als Erzählstil)
\end{itemize}

\chapter{Relevanz des Forschungsgegenstandes}

\begin{itemize}
    \item Generelle Begründung der Forschungsrelevanz über \emph{politics of presence} von \textcite{phillips_1998} und \textcite{pitkin_1972} - Repräsentanz von Frauen in Parlamenten. Vorbildfunktion und Veränderung durch Repräsentation.
    \item Sehr kurz nur das Thema \emph{gendered parliament} von \textcite{shelley_2011}
    \item Relevanz der Sprache über \textcite*{stahlberg_2001} und \textcite{stahlberg_2007} und auch \textcite{menegatti_2017}
    \item Problematik: nicht zu viele 
Nebenbastellen aufmachen. Hier nur wichtig: Repräsentation von Frauen für die Demokratie elementar wichtig. Sprache macht etwas mit Menschen: Reproduktion von Stereotypen und \enquote{Vergessen} bestimmter Personengruppen. Sprache kann sichtbar machen und Stereotypen brechen \parencite{sczesny_2016}.
    \item Es soll aber nicht einfach nur \enquote{gezählt} werden, sondern auch untersucht werdne, ob es bei den Themen Unterschiede gibt. Bezug wieder auf \emph{politics of presence}u\end{itemize}

\section{Derzeitiger Forschungsstand und relevante Studien}
\begin{itemize}
    \item Hier nur empirische Studien, nicht die Theorie neu aufmachen
    \item Sagen, auf welche Theorie sich die Erhebungen stützen
    \item \textcite{eagly_2002} stützt sich beispielsweise die auf die \emph{congruity theory} und konnte Nachweisen, dass Frauen eher Diskriminierung erfahren als männliche Kollegen
    \item Am nächsten zu unserer Arbeit ist \textcite{back_2014} sowie \textcite{back_2018}, allerdings hier die Problematik der \emph{hard} und \emph{soft topics}, welche nicht in der Arbeit wiederholt werden soll. Außerdem Problematik des \emph{topic modelings}
    \item Ebenfalls relevant ist die zusammenfassende Arbeit von Wängnerud \parencites*{wangnerud_2000}{wangnerud_2009} da \textcite{back_2014} darauf aufbaut.
\end{itemize}

\chapter{Übergeordnete Forschungsfrage und Konzeption der Arbeit}
\begin{itemize}
    \item Basierend auf der dargestellten Relevanz und des derzeitigen Forschungsstandes
    \item Vorstellung der übergeordneten Forschungsfrage
    \item Aufklärung, dass diese in drei Teile zerlegt wird
    \item Sprache, Themen und Unterbrechungen
\end{itemize}

\section{Hypothese Genderinklusive Sprache}

\section{Hypothese Themen in den Reden}

\section{Hypothese Unterbrechungen und Rückfragen in den Reden}

\section{Kontrollvariablen}
\begin{itemize}
    \item Klar, dass es nicht nur Sprache ist, die Auswirkungen zeigt
    \item x-zentrierte Untersuchung
    \item Erklärung, welche Kontrollvariablen untersucht werden
    \item Auch: Warum? Andere Studien, Erwartungen etc.
\end{itemize}

\chapter{Datengrundlage und Methodik}
\begin{itemize}
    \item Besonderheit der Protokolle des 19. Bundestags, Format und Hintergrund
\end{itemize}

\section{Vorgehen zur Auswertung der Bundestagsprotokolle}
\begin{itemize}
    \item Beispiele zur Protkollierung aus dem Bundestag
    \item R und die verwendeten Pakete
\end{itemize}

\section{Methodik Regression und automated topic modelling}

\subsection{Hintergrund stm}

\chapter{Ergebnisse und Auswertungen}

\section{Genderinklusive Sprache}

\section{Themen der Reden}

\section{Unterbrechungen und Rückfragen}

\chapter{Methodenkriktik und Forschungsausblick}

\chapter{Fazit}

\setstretch{1.0}
\printbibliography[title={Literaturverzeichnis}]

\end{document}