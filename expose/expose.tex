\documentclass[12pt,ngerman,]{article}
\usepackage[left=1in,top=1in,right=1in,bottom=1in]{geometry}
\newcommand*{\authorfont}{\fontfamily{phv}\selectfont}
\usepackage[]{mathpazo}


  \usepackage[T1]{fontenc}
  \usepackage[utf8]{inputenc}



\usepackage{abstract}
\renewcommand{\abstractname}{}    % clear the title
\renewcommand{\absnamepos}{empty} % originally center

\renewenvironment{abstract}
 {{%
    \setlength{\leftmargin}{0mm}
    \setlength{\rightmargin}{\leftmargin}%
  }%
  \relax}
 {\endlist}

\makeatletter
\def\@maketitle{%
  \newpage
%  \null
%  \vskip 2em%
%  \begin{center}%
  \let \footnote \thanks
    {\fontsize{18}{20}\selectfont\raggedright  \setlength{\parindent}{0pt} \@title \par}%
}
%\fi
\makeatother




\setcounter{secnumdepth}{0}



\title{Geschlechterunterschiede in den Reden des Deutschen Bundestages  }



\author{\Large Josef Holnburger\vspace{0.05in} \newline\normalsize\emph{Universität Hamburg}   \and \Large Gina Gabriela Görner\vspace{0.05in} \newline\normalsize\emph{Universität Hamburg}  }


\date{}

\usepackage{titlesec}

\titleformat*{\section}{\normalsize\bfseries}
\titleformat*{\subsection}{\normalsize\itshape}
\titleformat*{\subsubsection}{\normalsize\itshape}
\titleformat*{\paragraph}{\normalsize\itshape}
\titleformat*{\subparagraph}{\normalsize\itshape}





\newtheorem{hypothesis}{Hypothesis}
\usepackage{setspace}

\makeatletter
\@ifpackageloaded{hyperref}{}{%
\ifxetex
  \PassOptionsToPackage{hyphens}{url}\usepackage[setpagesize=false, % page size defined by xetex
              unicode=false, % unicode breaks when used with xetex
              xetex]{hyperref}
\else
  \PassOptionsToPackage{hyphens}{url}\usepackage[unicode=true]{hyperref}
\fi
}

\@ifpackageloaded{color}{
    \PassOptionsToPackage{usenames,dvipsnames}{color}
}{%
    \usepackage[usenames,dvipsnames]{color}
}
\makeatother
\hypersetup{breaklinks=true,
            bookmarks=true,
            pdfauthor={Josef Holnburger (Universität Hamburg) and Gina Gabriela Görner (Universität Hamburg)},
             pdfkeywords = {},  
            pdftitle={Geschlechterunterschiede in den Reden des Deutschen Bundestages},
            colorlinks=true,
            citecolor=blue,
            urlcolor=blue,
            linkcolor=black,
            pdfborder={0 0 0}}
\urlstyle{same}  % don't use monospace font for urls

% set default figure placement to htbp
\makeatletter
\def\fps@figure{htbp}
\makeatother

\usepackage[german]{babel}
\usepackage[german=quotes]{csquotes}
\usepackage{booktabs}
\usepackage{float}
\usepackage{etoolbox}
\AtBeginEnvironment{quote}{\singlespacing\small}


% add tightlist ----------
\providecommand{\tightlist}{%
\setlength{\itemsep}{0pt}\setlength{\parskip}{0pt}}

\begin{document}
	
% \pagenumbering{arabic}% resets `page` counter to 1 
%
% \maketitle

{% \usefont{T1}{pnc}{m}{n}
\setlength{\parindent}{0pt}
\thispagestyle{plain}
{\fontsize{18}{20}\selectfont\raggedright 
\maketitle  % title \par  

}

{
   \vskip 13.5pt\relax \normalsize\fontsize{11}{12} 
\textbf{\authorfont Josef Holnburger} \hskip 15pt \emph{\small Universität Hamburg}   \par \textbf{\authorfont Gina Gabriela Görner} \hskip 15pt \emph{\small Universität Hamburg}   

}

}








\begin{abstract}

    \hbox{\vrule height .2pt width 39.14pc}

    \vskip 8.5pt % \small 

\noindent Mit Feedback der vorangegangenen Sitzung des Forschungsseminars soll die
Forschungsfrage „Inwieweit unterscheiden sich die Redebeiträge und
Verhalten von weiblichen und männlichen Abgeordneten im 19. Deutschen
Bundestag bezüglich Häufigkeit, Thematik und Geschlechterneutralität``
untersucht werden. \newline Es soll untersucht werden, ob Frauen in
ihren Reden beispielsweise signifikant häufiger unterbrochen werden als
Männer, ob Frauen eher geschlechterinklusive Sprache verwenden und ob
signifikante Unterschiede in den Themen der Reden festzustellen sind,
welche sich auf das Geschlecht zurückführen lassen. Hierfür werden 5.965
Reden von den Mitgliedern des Deutschen Bundestages (MdB) anhand der
Bundestagsprotokolle ausgewertet.


    \hbox{\vrule height .2pt width 39.14pc}


\end{abstract}


\vskip 6.5pt


\noindent \doublespacing Eine Vielzahl an Studien beschäftigt sich mit den Repräsentations- und
Partizipationsunterschieden von Frauen und Männern in Parlamenten
(Blaxill/Beelen \protect\hyperlink{ref-blaxill_2016}{2016}: 416). Eine
umfassende Untersuchung von Geschlechterunterschieden in Bezug auf
Sprache, Thematik sowie Verhalten der Parlamentsmitglieder des Deutschen
Bundestages ist allerdings bis zu diesem Zeitpunkt noch ausstehend.

Anne Phillips gilt als eine der wichtigsten Vertreterinnen auf dem
Gebiet der parlamentarischen Representativität von Frauen. In ihrem Werk
\enquote{The Politics of Presence} analyisiert Phillips
(\protect\hyperlink{ref-phillips_1998}{1998}) unter anderem die
Bereitschaft von Frauen, als Parlamentskandidatin ausgewählt zu werden,
Ministerin zu werden und Wahlen zu gewinnen (vgl. Blaxill/Beelen
\protect\hyperlink{ref-blaxill_2016}{2016}: 416f.). Da Frauen im
täglichen Leben andere Erfahrungen sammeln als Männer -- insbesondere
bezüglich der Kindererziehung, Bildung sowie der Auswahl an Berufen und
der Spalting in bezahlte und unbezahlte Arbeit, Gewalterfahrungen sowie
sexuellen Belästigungen -- plädierte Philips für die notwendige
Repräsentation von Frauen im Parlament, um andere Frauen vertreten zu
können und diese Situationen sichtbar zu machen (vgl. Wängnerud
\protect\hyperlink{ref-wangnerud_2009}{2009}: 52).

\begin{quote}
\enquote{There are particular needs, interests, and concerns that arise
from women's experience, and these will be inadequately addressed in a
politics that is dominated by men. Equal rights to a vote have not
proved strong enough to deal with this problem; there must also be
equality among those elected to once} (Phillips
\protect\hyperlink{ref-phillips_1998}{1998}: 66)
\end{quote}

Die Theorie der \emph{politics of presence} (Phillips
\protect\hyperlink{ref-phillips_1998}{1998}) diente Wängnerud
(\protect\hyperlink{ref-wangnerud_2000}{2000}) als Ausgangspunkt für
ihre Forschung im Schwedischen Parlament (\enquote{Testing the Politics
of Presence, Women's Representation in the Swedish Riksdag}, 2000).
Wägnerud konzentriert sich hierbei auf die Repräsentation von Frauen und
prüft die Hypothese, dass weibliche Politikerinnen die Interessen von
Frauen stärker vertreten als männliche Politiker (Wängnerud
\protect\hyperlink{ref-wangnerud_2000}{2000}: 84). \enquote{{[}It is{]}
difficult to repudiate the conclusion that women's interes are primarily
represented by female politicians}
(\protect\hyperlink{ref-wangnerud_2000}{2000}: 84). Folglich tritt die
Zahl der Frauen im Parlament zunehmen in den Vordergrund der
wissenschaftlichen Auseinandersetzung und der Politik.

Bei der Forschung zu Frauen im Parlament wird generell zwischen einer
deskriptiven und einer substantiellen Repräsentation (in einigen Fällen
außerdem die symoblische Form der Repräsenttation) unterschieden. Bei
der substantiellen Repräsentation von Frauen handelt es sich um ein
bisher weniger wissenschaftlich erforschtes Feld als die Analyse der
deskriptiven Repräsentation (Wängnerud
\protect\hyperlink{ref-wangnerud_2009}{2009}: 59). Letzteres legt den
Schwerpunkt auf die Analyse der Anzahl von Frauen in representativen
Institutionen, ersteres untersucht hingegen die Auswirkung der Präsenz
von Frauen in Parlamenten (Coffe/Schnellecke
\protect\hyperlink{ref-coffe_2013}{2013}: 14; Wängnerud
\protect\hyperlink{ref-wangnerud_2009}{2009}: 52). Die zentrale Fragen
lauten hierbei:

\begin{quote}
\enquote{{[}W{]}hether the widely professed aspiration to feminise
democracy -- and in so doing to politically empower women -- is a matter
largely of symbolism or substance {[}\ldots{}{]}}
\enquote{{[}\ldots{}{]} wheter the priority should simply be to increase
the proportion of women MPs in Parliament {[}\ldots{}{]} or as Hanna
Pitkin argued in 1967, to represent minds as well as bodies.}
{[}blaxill\_2016, S. 413{]}
\end{quote}

Bereits Piktin (\protect\hyperlink{ref-pitkin_1972}{1972}) argumentierte
in ihrem Hauptwerk \enquote{The Concept of Representation}, dass der
Interessenbegriff in der Repräsentationsdebatte omnipräsent
(\enquote{ubiquitious}) ist (Wängnerud
\protect\hyperlink{ref-wangnerud_2000}{2000}: 69). Für Pitkin sind es
die Parliamentarierinnen, welche sich den Wünschen und Interessen, dem
Wohlergehen sowie Themen der Frauen widmen und diese vertreten (vgl.
Blaxill/Beelen \protect\hyperlink{ref-blaxill_2016}{2016}: 413; Pitkin
\protect\hyperlink{ref-pitkin_1972}{1972}). Nach Phillips Theorie der
\emph{politics of presence}
(\protect\hyperlink{ref-phillips_1998}{1998}) kann angenommen werden,
dass weibliche Politikerinnen die Interessen und Wünsche weiblicher
Bürger*innen repräsentieren können. Hierbei wird die deskripitve und
substantielle Representation verknüpft (Wängnerud
\protect\hyperlink{ref-wangnerud_2009}{2009}: 52) -- ein höherer
Frauenanteil in Parlamenten führt auch zu einer stärkeren Thematisierung
der Belange von Frauen.

~

Die Forschung von Celis et al.
(\protect\hyperlink{ref-celis_2008}{2008}) greift diese Annahmen
nachmals empirisch auf und untersucht, in welchem Ausmaß die Beürfnisse
und Interessen von Frauen gesteigert werden, wenn die Anzahl an Frauen
in politischen Entscheidungen zunimmt (vgl. auch Galligan/Meier
\protect\hyperlink{ref-galligan_2016}{2016}: 4). Ein einfacher Anstieg
der Anzahl von Frauen reicht jedoch nicht aus, um einen signifikanten
Einfluss zu erreichen. Vielmehr müssen weibliche Abgeordnete, in
Anlehnung an Caul (\protect\hyperlink{ref-caul_2001}{2001}) auch in
einflussreichen, wichtigen Positionen vertreten sein, um tatsächliche
substantielle Repräsentanz zu erreichen (Caul
\protect\hyperlink{ref-caul_2001}{2001}; vgl. auch Coffe/Schnellecke
\protect\hyperlink{ref-coffe_2013}{2013}: 14).

~

Es lässt sich resümieren, dass es in der Literatur unterschiedliche
Einschätzungen darüber gibt, welche Auswirkungen zu erwarten sind, wenn
die Zahl der Frauen im Parlament steigt (Wängnerud
\protect\hyperlink{ref-wangnerud_2009}{2009}: 59). Besonders
hervorzuheben ist hierbei die Ländervergleichsstudie von Bäck/Debus
(\protect\hyperlink{ref-back_2018}{2018}) in welcher die Autor*innen
sieben europäische Länder bezüglich der Themen und Redeanteilen von
weiblichen Abgeordneten untersuchen. Bäck/Debus
(\protect\hyperlink{ref-back_2018}{2018}) resümieren zwar, dass Frauen
in Parlamenten länderübergreifend seltener das Wort ergreifen, dies
allerdings nicht auf einen generell niedrigeren Anteil an Frauen in den
Parlamenten zurückzuführen ist. Frauen in Parteien mit einem geringen
Frauenanteil ergreifen laut der Untersuchung von Bäck und Debus sogar
öfter das Wort als Frauen in Fraktionen mit einem hohen Anteil an
weiblichen Mitgliedern (Bäck/Debus
\protect\hyperlink{ref-back_2018}{2018}: 17).

~

Vor diesem Hintergrund soll in dieser Forschungsarbeit neben dem Anteil
an Reden von Frauen im Deutschen Bundestag auch inhaltlich auf die Reden
eingegangen werden und auch das Verhalten der Abgeordneten auf
geschlechtsspezifische Besondernheiten hin untersucht werden. Die
zentrale Forschungsfrage lautet deshalb \emph{\enquote{Inwieweit
unterscheiden sich die Redebeiträge und Verhalten von weiblichen und
männlichen Abgeordneten im 19. Deutschen Bundestag bezüglich Häufigkeit,
Thematik und Geschlechterneutralität}}.

Während sich ein großer Teil der wissenschaftlichen Arbeiten aus den in
den vorherigen Absätzen aufgezeigten Perspektiven mit der Partizipation
und Repräsentation von Frauen in Parlamenten auseinandersetzt, soll sich
im folgenden auch auf geschlechterspezifische Unterschiede und
Reproduktion von Geschlechterungerechtigkeit durch auf Sprache gewidmet
werden.

\hypertarget{der-einfluss-von-sprache-auf-geschlechtergerechtigkeit}{%
\section{Der Einfluss von Sprache auf
Geschlechtergerechtigkeit}\label{der-einfluss-von-sprache-auf-geschlechtergerechtigkeit}}

Laut Menegatti/Rubini (\protect\hyperlink{ref-menegatti_2017}{2017}) ist
die Sprache eine der einflussreichsten Faktoren, wodurch Sexismus und
Geschlechterdiskriminierung gefördert und reproduziert werden
(\protect\hyperlink{ref-menegatti_2017}{2017}: 1). Sprache kann hierbei
insbesondere die sozialen Asymmetrien von Status und Macht zugunsten des
Mannes reprodizeren (ebd.). Es existier{[}t{]}en eivernehmliche Normen,
wonach der prototypische Mensch ein Mann ist, was sich in den Strukturen
vieler Sprachen widerspiegelt und darin verankert ist. Viele
grammatikalische und syntaktische Regeln sind dabie so aufgebaut, dass
weibliche Ausdrücke normalerweise von entsprechenden männlichen Formen
abgeleitet werden (ebd.).

Männliche Substantive und Pronomen werden beispielsweise häufig mit
einer generischen Funktion verwendet um sich sowohl auf Männer als auch
auf Frauen zu beziehen. Auf diese Weise verschwinden Frauen allerdings
aus der mentalen Repräsentationen (vgl. Menegatti/Rubini
\protect\hyperlink{ref-menegatti_2017}{2017}; Stahlberg/Sczesny
\protect\hyperlink{ref-stahlberg_2001}{2001}). Maskuline Generika lassen
Leser*innen und Hörer*innen mehr in männlichen als weiblichen
Personenkategorien denken (Sczesny et al.
\protect\hyperlink{ref-sczesny_2016}{2016}: 2; Stahlberg et al.
\protect\hyperlink{ref-stahlberg_2007}{2007}).

\begin{quote}
\enquote{Given that language not only reflects stereotypical beliefs but
also affects recipients' cognition and behavior, the use of expressions
consistent with gender stereotypes contributes to transmit and reinforce
such belief system and can produce actual discrimination against women}.
(Menegatti/Rubini \protect\hyperlink{ref-menegatti_2017}{2017}: 2)
\end{quote}

Die Verwendung von \emph{gender-fail linguistic} (GFL) kann diese
negativen Auswirkungen effektiv verhindern und Geschlechtergerechtigkeit
fördern (Menegatti/Rubini \protect\hyperlink{ref-menegatti_2017}{2017}:
1). GFL wurde unter anderem im Rahmen eines umfassenden Versuchs zur
Verringerung von Stereotypen und Diskriminierung in der Sprache
eingeführt (Sczesny et al. \protect\hyperlink{ref-sczesny_2016}{2016}:
2). Feminiserung und Neutralisierung sowie die Kombination beider sind
die bevorzugten Strategien einer geschlechtergerechten oder
geschlechterneutralen Sprache. Auch die Verwendung von Wortpaaren
(Lehrerinnen und Lehrer) zählt ebenso zu den Ausdrucksformen der GFL.
Neben Wortpaaren sind auch geschlechtsneutrale Formen (Studierende statt
Student) mögliche GFL-Ausdrucksformen (Sczesny et al.
\protect\hyperlink{ref-sczesny_2016}{2016}: 2).

Aus den bereits genannten Problematiken ist es notwendig, die
Sprachgewohnheit dahingehend zu ändern, GFL umfassend zu etablieren, um
Vorurteile bezüglich der Geschlechter zu reduzieren und eine
Reproduktion von Stereotypen zu vermeiden. Die Verwendung von
geschlechtergerechter Ausdrücke anstelle von maskulinen Generika ist für
den Abbau von Geschlechtervoreingenommenheit und die Förderung der
Gleichstellung der Geschlechter laut Menegatti und Rubini unabdingbar
(\protect\hyperlink{ref-menegatti_2017}{2017}: 3).

Die Umsetzung und Etablierung von GFL hat in verschiedenen Ländern
bisher unterschiedliche Stadien erreicht und wird beispielsweise von der
UNESCO und der Europäischen Kommission empfohlen und in deren Dokumenten
angewandt (Sczesny et al. \protect\hyperlink{ref-sczesny_2016}{2016}:
4).

\begin{quote}
\enquote{{[}\ldots{}{]} language does not merely reflect the way we
think: it also shapes our thinking. If words and expressions that imply
that women are inferior to men are constantly used, that assumption of
inferiority tends to become part of our mindset; hence the need to
adjust our language when our ideas evolve.} (UNESCO 2011)
\end{quote}

Auf Basis der Argumentation von Wängnerud
(\protect\hyperlink{ref-wangnerud_2000}{2000}, @wangnerud\_2009) wird
erwartet, dass sich vor allem Frauen über die Ausgrenzungserfahrung
durch Sprache bewusst sind oder entsprechende Erfahrungen bereits
gemacht wurden und entsprechend eher eine geschlechterneutrale und
geschlechtergerechte Sprache anwenden als Männer.

\textbf{Hypothese 1:} \emph{Frauen verwenden in ihren Reden häufiger
gender-fair language als Männer}

\hypertarget{geschlechterbezogene-verhaltensunterschiede-im-bundestag}{%
\section{Geschlechterbezogene Verhaltensunterschiede im
Bundestag}\label{geschlechterbezogene-verhaltensunterschiede-im-bundestag}}

Die von Erikson und Josefsson durchgeführte Befragung zum
Arbeitsumwangefeld von schwedischen Abgeordneten konnte zeigen, dass
weibliche Abgeordnete mehr Stress und Druck in ihrem Umfeld ausgesetzt
sind und häufiger Opfer negativer Behandlungungen im Parlament sind als
männliche Abgeordnete (Erikson/Josefsson
\protect\hyperlink{ref-erikson_2018}{2018}). Sie werden eher in den
Debatten unterbrochen, sind eher Opfer sexistischer Witze und ihre
Äußerlichkeiten wird häufiger negativ kommentiert als die Erscheinung
männlicher Kollegen (ebd., S. 13).

Diese Mehrbelastung kann laut Erikson und Josefsson zu höheren
\enquote{Kosten} für Frauen bezüglich des parlamentarischen Engagements
führen -- sie müssen mit mehr negativen Erfahrungen rechnen als Männer,
entsprechend könnte dies auch ein Grund für eine geringere Beteiligung
von Frauen in Parlamentsdebatten oder generell für die Partizipation von
Frauen in poitischen Insitution darstellen (vgl. Erikson/Josefsson
\protect\hyperlink{ref-erikson_2018}{2018}; vgl. Bäck et al.
\protect\hyperlink{ref-back_2014}{2014}).

Es wird erwartet, dass ein solches negatives Verhalten gegenüber Frauen
auch im Deutschen Bundestag zu beobachten ist. Entsprechend soll
untersucht werden, wie häufig Abgeordnete in ihren Reden \emph{negativ}
unterbrochen werden. Entsprechend sollen Reden bezüglich der Anzahl an
Zurufen, Gegenrufen, Widerspruch und Lachen über den*die Redner*in
untersucht werden.\footnote{Diese Unterbrechungen werden in den
  Bundestagsprotokollen vermerkt. \emph{Lachen} kann hier als
  \emph{auslachen} oder \emph{lachen über den Inhalt} verstanden werden
  -- das Lachen beispielsweise über einen Witz wird in
  Bundestagsprotokollen als \emph{Heiterkeit} protokolliert.
  \emph{Positive} Unterbrechungen wären beispielsweise Beifall.}

\textbf{Hypothese 2a:} \emph{Frauen werden häufiger als Männer während
der Rede negativ Unterbrochen}

~

Eine weitere \emph{negative} Unterbrechung der Render*innen können
Rückfragen darstelen -- allerdings ist hier eine eindeutige negative
Intention nicht anzunehmen. Entsprechend sollen Rückfragen gesondert
untersucht werden. Auf Basis der Arbeiten von Brescoll
(\protect\hyperlink{ref-brescoll_2011}{2011}) und Eagly/Karau
(\protect\hyperlink{ref-eagly_2002}{2002}) wird allerdings erwartet,
dass Frauen bei Themen, welche stereotypisch eher nicht mit Frauen
verbunden werden, häufiger Inkompetenz unterstellt wird als männlichen
Abgeordneten. Dies kann sich in der Anzahl an Rückfragen äußern, welche
während einer Rede gestellt werden.

\textbf{Hypothese 2b:} \emph{Frauen erfahren häufiger Rückfragen in
ihren Reden, wenn wenige andere Frauen zu diesem Thema gesprochen
haben}\footnote{Diese Hypothese kann nur in Verbindung mit der
  nachfolgenden Hypothese zur Thematisierung in den Reden überprüft
  werden.}

~

\hypertarget{geschlechterspezifische-themen-der-reden}{%
\section{Geschlechterspezifische Themen der
Reden}\label{geschlechterspezifische-themen-der-reden}}

Bäck et al. (\protect\hyperlink{ref-back_2014}{2014}) konnten in ihrer
Untersuchung des schwedischen Parlaments nachweisen, dass bezüglich der
Themen der Reden Geschlechterunterschiede festzustellen sind. Männer
sprechen laut Bäck et. al.~im schwedischen \emph{Riksdag} häufiger zu
sogenanntenn \emph{hard policies} (etwa Wirtschaft, Energie,
Infrastruktur). Bei sogenannten \emph{soft policies} (Erziehung, Soiale
Sicherung, Bildung) ist hingegen kein Unterschied bezüglich des
Redeanteils und Geschlecht der Abgeordneten festzustellen (Bäck et al.
\protect\hyperlink{ref-back_2014}{2014}: 514f.).

Die von Bäck et. al.~vorgenommene Klassifiierung soll in unserer
Forschung nicht vorgenommen werden, da hier die Gefahr besteht, in der
Gesellschaft vorhandene Stereotypen zu reproduzieren, indem eine
klassifizierung in vermeintliche geschlechtsspezifische Themen
vorgenommen wird.

Dennoch wird, auf Basis der Argumentation von Wängnerud
(\protect\hyperlink{ref-wangnerud_2000}{2000}, @wangnerud\_2009) und der
Arbeit von Bäck et al. (\protect\hyperlink{ref-back_2014}{2014})
untersucht werden, ob eine geschlechtsspezifische, unterschiedliche
Thematisierung in den Reden feststellbar ist.

\textbf{Hypothese 3:} \emph{Es ist unterschiedliche,
geschlechtsspezifische Thematisierungen in den Reden der Abgeordneten
feststellbar.}

\hypertarget{weitere-erganzende-untersuchungen}{%
\section{Weitere ergänzende
Untersuchungen}\label{weitere-erganzende-untersuchungen}}

Da die derzeitige Legislaturperiode noch nicht Gegenstand der
Untersuchungen der Partizipation von Frauen waren, sollen auch weitere
Metriken erhoben werden, etwa die Anzahl an Reden von Frauen nach
Fraktion im Vergleich zu anteiligen Sitzungen von Frauen nach Fraktion.
Da Bäck/Debus (\protect\hyperlink{ref-back_2018}{2018}) in einem
Ländervergleich allerdings keine Verbindungen zwischen der Anzahl an
Frauen in den Fraktionen und Anzahl der Reden in Parlamenten von Frauen
aufzeigen konnte, soll dies nur als Ergänzung zu den Untersuchungen
aufgenommen werden.

\pagebreak

\hypertarget{datengrundlage-und-operationalisierung}{%
\section{Datengrundlage und
Operationalisierung}\label{datengrundlage-und-operationalisierung}}

Seit der 19. Wahlperiode liegen die Protokolle des Deutschen Bundestags
im TEI-Format (Text Encoding Initative)\footnote{\url{http://www.tei-c.org/guidelines/}}
als XML-Dateien (Extensible Markup Language) vor. Abgerufen werden
können sie über das \emph{Open Data Portal} des Deutschen
Bundestags.\footnote{\url{https://www.bundestag.de/service/opendata}} In
Verbindung mit den biografischen Daten aller
Bundestagsabgeordneten\footnote{\url{https://www.bundestag.de/blob/472878/e207ab4b38c93187c6580fc186a95f38/mdb-stammdaten-data.zip}}
können Aussagen über Anzahl und Inhalte der Reden nach Geschlecht,
Alter, Fraktion und beispielsweise Anzahl der Bundestagsperioden
getroffen werden.

Insgesamt liegen 5.965 Reden\footnote{Es liegen auch Reden der
  Regierungsmitglieder und Reden von Gästen vor, diese werden allerdings
  nicht ausgewertet, da hierbei keine Angaben zum Geschlecht der
  Redner*innen durch den Bundestag bereitsgestellt werden und auch nur
  die Bundestagsdebatte und nicht die Berichterstattung durch die
  Regierung ausgewertet werden soll.} von Bundestagsabgeordneten von
Abgeordneten des Deutschen Bundestages mit Anmerkungen durch die
Stenograf*innen vor. Diese Anmerkungen ermöglichen die
Operationalisierung der Hypothesen -- so wird bei einer Rede auch ein
Zwischenruf protokolliert und auch der*die Zwischenrufer*in im Protokoll
vermerkt (solange der Zwischenruf zugeordnet werden kann). Auch Angaben
über die Tagesordnungspunkte (TOPs), zu denen gesprochen wird, sind im
Protokoll vermerkt.

Die Daten werden mittels der Programmiersprache R (R Core Team
\protect\hyperlink{ref-rcoreteam_2018}{2018}) und den Packeten
\emph{rvest} (Wickham \protect\hyperlink{ref-wickham_2016}{2016}),
\emph{xml2} (Wickham et al. \protect\hyperlink{ref-wickham_2018}{2018})
und der Packetsammlung \emph{tidyverse} (Wickham
\protect\hyperlink{ref-wickham_2017}{2017}) aufbereitet und ausgewertet.

\begin{table}[!h]

\caption{\label{tab:unnamed-chunk-1}Anteil der Reden im 19. Bundestag nach Geschlecht der Abgeordneten}
\centering
\begin{tabular}{lr}
\toprule
Geschlecht & Reden\\
\midrule
männlich & 4149\\
weiblich & 1816\\
\bottomrule
\end{tabular}
\end{table}

\hypertarget{forschungsdesign-und-methodik}{%
\section{Forschungsdesign und
Methodik}\label{forschungsdesign-und-methodik}}

Wie die Fragestellung \enquote{Inwieweit unterscheiden sich die
Redebeiträge und Verhalten von weiblichen und männlichen Abgeordneten im
19. Deutschen Bundestag bezüglich Häufigkeit, Thematik und
Geschlechterneutralität} bereits suggeriert, wird davon ausgegangen,
dass die Variable \emph{Geschlecht} einer von vielen Faktoren ist,
welche das Verhalten, die Sprache und den Inhalt der Bundestagsreden
beeinflussen. Entsprechend handelt es sich bei dieser Untersuchung um
eine x-zentriertes Forschungsdesign, in welchem auch weitere Faktoren,
welche auf die jeweilige abhängige Variable einwirken, untersucht werden
sollen (vgl. Ganghof \protect\hyperlink{ref-ganghof_2005}{2005}: 3f.).

~

Die Überprüfung der \textbf{Hypothesen 1, 2a und 2b} sollen mittels
Regressionsanalysen untersucht werden. Neben dem Geschlecht des*der
Redner*in sollen folgende Faktoren als Kontrollvariablen erhoben und
untersucht werden:

\begin{itemize}
\tightlist
\item
  \textbf{der*die Rednerin ist Mitglied der Oppositions- bzw.
  Regierungsfraktion}
\end{itemize}

Es wird erwartet, dass Mitglieder der Regierungsfraktion deutlich
häufiger mit Rückfragen konfrontiert werden, da sie eine stärkere
Einbindung in den Legislativprozess erfahren und die Kontrolle der
Regierung zu den Kernaufgaben der Opposition zählen.

\begin{itemize}
\tightlist
\item
  \textbf{der*die Redner*in ist Mitglied des Fraktionsvorsitzes}
\end{itemize}

Es wird erwartet, dass Mitglieder des Vorsitzes der Fraktionen
beispielsweise häufiger unterbrochen werden. Zum einen, da Reden, welche
auf Antrag der Opposition länger gehalten werden, häufig von den
Fraktionsvorsitzenden gehalten werden, aber auch, da diese Reden häufig
bei volleren Plenarsälen gehalten werden.

\begin{itemize}
\tightlist
\item
  \textbf{der*die Redern*in ist Mitglied der Fraktion
  \enquote{Alternative für Deutschland}}
\end{itemize}

Hier wird erwartet, dass Abgeordnete der AfD-Fraktion deutlich seltener
GFL nutzen, wohingehend die Anzahl der Unterbrechungen deutlich höher
erwartet wird. Populistische Parteien setzen sich unter anderem durch
provokante Thesen und ein provokantes Auftreten von anderen Parteien ab
(vgl. Decker \protect\hyperlink{ref-decker_populismus_2006}{2006};
Priester \protect\hyperlink{ref-priester_2012}{2012}). Dies lässt
vermuten, dass mehr Zwischenrufe von anderen Abgeordneten in den Reden
der AfD-Mitglieder geäußert werden.

\begin{itemize}
\tightlist
\item
  \textbf{die Anzahl der bisherigen Bundestagsmandate des*der Redner*in}
\end{itemize}

Bei dieser Variable wird kein Einfluss auf die abhängigen Variablen
(Nutzung von GFL, Anzahl der Unterbrechungen, Anzahl der Rückfragen)
erwartet, sie dient der Überprüfung der Validität der Methoden.

\begin{itemize}
\tightlist
\item
  \textbf{das Alter des*der Redner*in}
\end{itemize}

Hier wird keine Einfluss auf die abhängigen Variablen erwartet. Diese
Variable dient der Überprüfung der Validität der Methode.

~

Die \textbf{Hypothesen 1, 2a und 2b} werden über eine Regressionsanalyse
überprüft. Die Variable Geschlecht wird hierbei als Dummy-Variable
kodiert. Der Einfluss der Kontrollvariablen etwa auf die abhängige
Variable \emph{Anzahl der Unterbrechungen} wird ebenfalls überprüft. Da
es sich bei den abhängigen Variablen um Zähldaten handelt, wird
vermutlich eine negative Binomialregression angewandt (welche auch bei
Bäck et al. (\protect\hyperlink{ref-back_2014}{2014}) angewandt wurde)
oder eine OLS-Regression untersucht (angewandt bei Coffe/Schnellecke
(\protect\hyperlink{ref-coffe_2013}{2013})).

\textbf{Hypothese 3} zur Untersuchung der Themen in den Reden der
Abgeordneten soll über ein automatisiertes \emph{Topic Modeling}
untersucht werden. Anders als bei Bäck et al.
(\protect\hyperlink{ref-back_2014}{2014}), welche die Themen der Reden
der Abgeordneten anhand der vorangegangen Reden von Minister*innen
untersucht hat, soll durch ein automatisiertes \emph{Topic Modeling}
einerseits einen geschlechterstereotypen Bias verhindern aber auch eine
genauere Zuordnung an Themen ermöglichen. Automatisierte \emph{Topic
Modelings} wurden bereits auf Reden des Europaparlaments angewandt, um
Agenda-Trends zu identifizieren (vgl. Greene/Cross
\protect\hyperlink{ref-greene_2016}{2016}) und werden auch in der
vergangenen Studie von Bäck und Debus zur Identifikation von
geschlechtsbezogener \emph{Topics} vorgeschlagen (Bäck/Debus
\protect\hyperlink{ref-back_2018}{2018}: 18).

Als automatisiertes \emph{Topic Modeling Framework} wird das von der
University of Cambridge ausgezeichnete\footnote{\url{https://www.cambridge.org/core/membership/spm/about-us/awards/statistical-software-award/statistical-software-award-18}}
\emph{Structural Topic Modeling} von Roberts et al.
(\protect\hyperlink{ref-roberts_2018}{2018}) angewendet. Dies ermöglicht
es, auch innerhalb von Reden mehrere Inhalte zu identifizieren und
erlaubt so ein genaueres Vorgehen bei der Untersuchung möglicher
geschlechterbezogener Themen. In unserem Datensatz sind 6823 Reden
enthalten.

\pagebreak
\section{Literatur- und Quellverzeichnis}
\setlength{\parindent}{-0.2in}
\setlength{\leftskip}{0.2in}
\setlength{\parskip}{8pt}
\vspace*{-0.2in}
\setstretch{1.0}

\noindent

\hypertarget{refs}{}
\leavevmode\hypertarget{ref-back_2018}{}%
Bäck, Hanna/Debus, Marc (2018): When Do Women Speak? A Comparative
Analysis of the Role of Gender in Legislative Debates. In:
\emph{Political Studies}, DOI:
\href{https://doi.org/10.1177/0032321718789358}{10.1177/0032321718789358}.
Text abrufbar unter: \url{https://doi.org/10.1177/0032321718789358}.

\leavevmode\hypertarget{ref-back_2014}{}%
Bäck, Hanna/Debus, Marc/Müller, Jochen (2014): Who Takes the
Parliamentary Floor? The Role of Gender in Speech-Making in the Swedish
"Riksdag". In: \emph{Political Research Quarterly}, 67 (3), 504--518.

\leavevmode\hypertarget{ref-blaxill_2016}{}%
Blaxill, Luke/Beelen, Kaspar (2016): A Feminized Language of Democracy?
The Representation of Women at Westminster Since 1945. In:
\emph{Twentieth Century British History}, 27 (3), 412--449.

\leavevmode\hypertarget{ref-brescoll_2011}{}%
Brescoll, Victoria L. (2011): Who Takes the Floor and Why: Gender,
Power, and Volubility in Organizations. In: \emph{Administrative Science
Quarterly}, 56 (4), 622--641.

\leavevmode\hypertarget{ref-caul_2001}{}%
Caul, Miki (2001): Political Parties and the Adoption of Candidate
Gender Quotas: A Cross-National Analysis. In: \emph{The Journal of
Politics}, 63 (4), 1214--1229.

\leavevmode\hypertarget{ref-celis_2008}{}%
Celis, Karen/Childs, Sarah/Kantola, Johanna/Krook, Mona Lena (2008):
RETHINKING WOMEN'S SUBSTANTIVE REPRESENTATION. In:
\emph{Representation}, 44 (2), 99--110.

\leavevmode\hypertarget{ref-coffe_2013}{}%
Coffe, H./Schnellecke, K. (2013): Female Representation in German
Parliamentary Committees. 1972--2009. Paper Prepared for Presentation at
the European Consortium for Political Research General Conference.
Bordeaux. Text abrufbar unter:
\url{https://ecpr.eu/filestore/paperproposal/24876915-576b-42af-a5e8-55630fb57038.pdf}.

\leavevmode\hypertarget{ref-decker_populismus_2006}{}%
Decker, Frank (2006): Populismus: Gefahr Für Die Demokratie Oder
Nützliches Korrektiv? VS Verlag für Sozialwissenschaften. Text abrufbar
unter: \url{//www.springer.com/de/book/9783531145372} (Zugriff am
3.3.2018).

\leavevmode\hypertarget{ref-eagly_2002}{}%
Eagly, Alice H./Karau, Steven J. (2002): Role Congruity Theory of
Prejudice Toward Female Leaders. In: \emph{Psychological Review}, 109
(3), 573--598.

\leavevmode\hypertarget{ref-erikson_2018}{}%
Erikson, Josefina/Josefsson, Cecilia (2018): The Legislature as a
Gendered Workplace: Exploring Members of Parliament's Experiences of
Working in the Swedish Parliament: In: \emph{International Political
Science Review}, DOI:
\href{https://doi.org/10.1177/0192512117735952}{10.1177/0192512117735952}.
Text abrufbar unter:
\url{https://journals.sagepub.com/doi/full/10.1177/0192512117735952}.

\leavevmode\hypertarget{ref-galligan_2016}{}%
Galligan, Yvonne/Meier, Petra (2016): The Gender-Sensitive Parliament:
Recognising the Gendered Nature of Parliaments. Präsentiert auf: IPSA
World Congress, Juli 2016, Poznan, Poland.

\leavevmode\hypertarget{ref-ganghof_2005}{}%
Ganghof, Steffen (2005): Kausale Perspektiven in der vergleichenden
Politikwissenschaft: X-zentrierte und Y-zentrierte Forschungsdesigns.
In: Kropp, Sabine/Minkenberg, Michael (Hrsg.), Vergleichen in der
Politikwissenschaft. Wiesbaden: VS Verlag für Sozialwissenschaften,
76--93.

\leavevmode\hypertarget{ref-greene_2016}{}%
Greene, Derek/Cross, James P. (2016): Exploring the Political Agenda of
the European Parliament Using a Dynamic Topic Modeling Approach. In:
Text abrufbar unter: \url{http://arxiv.org/abs/1607.03055} (Zugriff am
11.1.2019).

\leavevmode\hypertarget{ref-menegatti_2017}{}%
Menegatti, Michela/Rubini, Monica (2017): Gender Bias and Sexism in
Language. In: \emph{Oxford Research Encyclopedia of Communication}, DOI:
\href{https://doi.org/10.1093/acrefore/9780190228613.013.470}{10.1093/acrefore/9780190228613.013.470}.
Text abrufbar unter:
\url{http://oxfordre.com/view/10.1093/acrefore/9780190228613.001.0001/acrefore-9780190228613-e-470}.

\leavevmode\hypertarget{ref-phillips_1998}{}%
Phillips, Anne (1998): The Politics of Presence. Oxford University
Press.

\leavevmode\hypertarget{ref-pitkin_1972}{}%
Pitkin, Hanna Fenichel (1972): The Concept of Representation. 1.
paperback ed., {[}Nachdr.{]}. Berkeley, Calif.: Univ. of California
Press. OCLC: 838205515.

\leavevmode\hypertarget{ref-priester_2012}{}%
Priester, Karin (2012): Wesensmerkmale Des Populismus. In: \emph{Aus
Politik und Zeitgeschichte}, 62. Jahrgang (5--6/2012), 3--9.

\leavevmode\hypertarget{ref-rcoreteam_2018}{}%
R Core Team (2018): R: A Language and Environment for Statistical
Computing. Vienna, Austria: R Foundation for Statistical Computing. Text
abrufbar unter: \url{https://www.R-project.org/}.

\leavevmode\hypertarget{ref-roberts_2018}{}%
Roberts, Margaret E./Stewart, Brandon M./Tingley, Dustin (2018): Stm: R
Package for Structural Topic Models. Text abrufbar unter:
\url{http://www.structuraltopicmodel.com}.

\leavevmode\hypertarget{ref-sczesny_2016}{}%
Sczesny, Sabine/Formanowicz, Magda/Moser, Franziska (2016): Can
Gender-Fair Language Reduce Gender Stereotyping and Discrimination? In:
\emph{Frontiers in Psychology}, 7.

\leavevmode\hypertarget{ref-stahlberg_2001}{}%
Stahlberg, Dagmar/Sczesny, Sabine (2001): Effekte des generischen
Maskulinums und alternativer Sprachformen auf den gedanklichen Einbezug
von Frauen. In: \emph{Psychologische Rundschau}, 52 (3), 131--140.

\leavevmode\hypertarget{ref-stahlberg_2007}{}%
Stahlberg, D./Braun, F./Irmen, L./Sczesny, S. (2007): Representation of
the Sexes in Language. In: Fiedler, K. (Hrsg.), Social Communication. A
Volume in the Series Frontiers of Social Psychology. NewYork: NY:
Psychology Press, 163--187.

\leavevmode\hypertarget{ref-wangnerud_2000}{}%
Wängnerud, Lena (2000): Testing the Politics of Presence: Women's
Representation in the Swedish Riksdag. In: \emph{Scandinavian Political
Studies}, 23 (1), 67--91.

\leavevmode\hypertarget{ref-wangnerud_2009}{}%
Wängnerud, Lena (2009): Women in Parliaments: Descriptive and
Substantive Representation. In: \emph{Annual Review of Political
Science}, 12 (1), 51--69.

\leavevmode\hypertarget{ref-wickham_2016}{}%
Wickham, Hadley (2016): Rvest: Easily Harvest (Scrape) Web Pages. Text
abrufbar unter: \url{https://CRAN.R-project.org/package=rvest}.

\leavevmode\hypertarget{ref-wickham_2017}{}%
Wickham, Hadley (2017): Tidyverse: Easily Install and Load the
'Tidyverse'. Text abrufbar unter:
\url{https://CRAN.R-project.org/package=tidyverse}.

\leavevmode\hypertarget{ref-wickham_2018}{}%
Wickham, Hadley/Hester, James/Ooms, Jeroen (2018): Xml2: Parse XML. Text
abrufbar unter: \url{https://CRAN.R-project.org/package=xml2}.




\newpage
\singlespacing 
\end{document}
